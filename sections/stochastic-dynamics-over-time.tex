\section{Dynamics Over Time}
\subsection{Law of Decreasing Labor Content}
	Assume for simplicity that the capital of a firm $f_i$ is unchanging between a particular pair of periods $T_1,T_2$. Suppose that the firm found a way to reduce it's unit costs (that is, reduce the dollar cost of producing a unit of it's output), whereas all competing firms (that is, firms producing the same product) continue to produce at the conventional standard costs. Then there is no reason to assume that the price of this product once sold will differ from what it was before; that is to say, it is extremely likely that either
\begin{itemize}
	\item[(1)] The firm produces the usual amount of product at the reduced unit costs. In this case $\Pr(f_i)$ will very likely remain the same as in the previous period, while $I(f_i)-V(f_i)$ will shrink (that's simply the hypothesis) and equivalently $K(f_i)$ will also be able to shrink, so that the profit rate $R(f_i)$ increases
	\item[(2)] The firm produces invests the same or more capital in production at the reduced unit costs, i.e. $K(f_i)$ either remains the same or increases, and equivalently $I(f_i)-V(f_i)$ remains the same or increases. $Pr(f_i)$ however will increase to remain higher than the increase in $K(f_i)$, and expressing the rate of profit by
	\[ R = \frac{Pr}{K}-1 \]
makes it cleat that the rate of profit increases. 
\end{itemize} 
A third option would be that the firm intentionally sells their product at a lower price, in which case their rate of profit can remain the same, while nonetheless squeezing their competitors and pushing them out of the market. In any case, it is clear that in a capitalist economy, capitalist firms will seek to reduce the unit costs of production. There are four ways in which this can be pursued:
\begin{itemize}
	\item[(1)] Reduce the workers' rates of pay
	\item[(2)] Beat down the prices it pays to suppliers of non-labour inputs
	\item[(3)] Reduce the direct labour time spent per unit of output, either by speeding up the labour process or by "rationalizing" it.
	\item[(4)] Replace existing non-labour inputs by others which it can buy more cheaply. 
\end{itemize} 
What I'm concerned with is the way in which this incentive structure of a capitalist economy will over time reduce the labour content of an arbitrary commodity. Towards showing this, items (1) and (2) have no influence. Firms will pursue these options, without a doubt, but will find themselves hindered by direct resistance of the workers and other supplying firms, forcing the firm in question to turn to (3) and (4) more often than these two. \par 
The effect of strategy (3) is direct - if this is accomplished, then the labour content of the commodity in question will fall, and competing firms will find themselves forced to adopt similar organizational schemes, resulting in an overall fall of the labour content of the commodity in general. \par 
This leaves us at option (4), which is by far the most commonly pursued method of reducing costs. Unlike (3), there is no direct reason to assume that this will result in a reduction in labour contents, even as it is employed repeatedly across the economy. This is what we wish to show. Suppose that a firm uses up a particular bundle of commodities $\mathbb{B}_1$ of non-labour inputs to produce one unit of output. This is replaced by a new bundle $\mathbb{B}_2$ which is a completely different set of materials, but with the property that $\pi(\mathbb{B}_1) > \pi(\mathbb{B}_2)$. For simplicity we'll write $\pi_1$ and $\pi_2$ in place of $\mathbb{B}_1$ and $\mathbb{B}_2$. Define the \textbf{cheapening factor} $c$ by
	\[ c = \frac{\pi_1}{\pi_2} \]
We will also define the \textbf{labour content ratio}, abbreviated lcr, as 
	\[ h = \frac{\lambda_1}{\lambda_2} \]
where of course $\lambda_1 = \Lambda(\mathbb{B}_1)$ and $\lambda_2 = \Lambda(\mathbb{B}_2)$. We know that $c > 1$, and we wish to inspect what $h$ will be. Of course, the ratios $\psi_1 = \frac{\pi_1}{\lambda_1}$ and $\psi_2 = \frac{\pi_2}{\lambda_2}$ are the specific prices of these commodity bundles. Writing $h$ in terms of the specific prices gives us a relation between $h$ and $c$:
\begin{align}
	h = \frac{\lambda_1}{\lambda_2} \frac{c}{c} = c \frac{\lambda_1}{\lambda_2} \frac{\pi_2}{\pi_1} = c \frac{\lambda_1}{\pi_1} \frac{\pi_2}{\lambda_2} = c \frac{1}{\psi_1} \psi_2 = c \frac{\psi_2}{\psi_1}
\end{align}
Towards replacing this particular instance with a general one involving random variables, note that from the perspective of the firm, the specific prices of commodities are invisible. They only see prices in dollars, and so the resulting change in specific prices might as well be considered random. Suppose the firm $f_i$ uses the bundle $\mathbb{I}$ consists of the commodities $I_1,I_2,...,I_M$ (re-indexed from market space). Each of these has a specific price $\Psi(I_1) = \psi_1, \Psi(I_2) = \psi_2$,...,$\Psi(I_M) = \psi_M$, as well as a labour content $\lambda_1,...,\lambda_M$. The capital $M$ is to indicate that $M$, the number of commodities purchased as non-labour inputs, is itself a random variable on firm space. This is going to make things complicated, however, so we will hold $m$ as non-random by fixing it to be the maximum number of inputs bought by \textit{any} firm during the period in question, and where a particular firm has only bought $j < m$ many commodities, we simply let $\lambda_{j+1}= \lambda_{j+2} = \ldots... = \lambda_m = 0$.  What we are interested in when it comes to specific price is the ratio of total price to total labour content. The total labour content is obviously $\sum \lambda_i$. Meanwhile, the total price can be written $\sum \lambda_i \psi_i$, since this product is precisely $\pi_i$. Thus we define the \textbf{aggregate specific price} of the $i^{th}$ firm to be
 \[ \bar{\Psi}(f_i) = \frac{\sum_{i=1}^m \lambda_i \psi_i}{\sum_{i=1}^M \lambda_i} \]
The numerator in particular here is itself a random variable representing the total price of these inputs, which we will call the \textbf{aggregate price}, and denote $\bar{\Pi}$. The denominator meanwhile is of course the \textbf{aggregate labour content}, denoted $\bar{\Lambda}$. Thus we can see that another way to express aggregate specific price is with the identity 
\[\bar{\Psi} = \frac{\bar{\Pi}}{\bar{\Lambda}} \]
In the following analysis, we will actually be considering random variables over the product of two firm spaces for two different periods $T_1$ and $T_2$. In each period we have commodity bundles $\mathbb{I}_1$ and $\mathbb{I}_2$. For each of these, we have two aggregate prices $\bar{\Pi}_1$ and $\bar{\Pi}_2$, and two aggregate specific prices $\bar{\Psi}_1$, $\bar{\Psi}_2$. From these, and from the expression derived earlier more informally relating the lcr to the cheapening factor by way of specific prices, we can define the cheapening factor $C$ and the lcr as random variables:
\[ C = \frac{\bar{\Pi}_1}{\bar{\Pi}_2} \]
\[ H = C \frac{\bar{\Psi}_1}{\bar{\Psi}_2} = \frac{\bar{\Lambda}_1}{\bar{\Lambda}_2} \]
Where the equality in the expressions of $H$ follows in exactly the same way as before, via the fact that aggregate specific price is the ratio of aggregate price to aggregate labour content. \par 
Consider instead of $H$, the log of $H$:
\[ \log(H) = \log(C) + \log(\bar{\Psi}_1) - \log(\bar{\Psi}_2) \]
Note of course that $H > 1$ if and only if $\log(H) > 0$. Taking the expected value of both sides and applying linearity we obtain
\[ E(\log(H)) = E(\log(C)) + E(\log(\bar{\Psi}_1) - E\log(\bar{\Psi}_2)) \]
Now consider $\Psi_1$ and $\Psi_2$ more closely. It might not be the case that the market transactions corresponding to the bundles $\mathbb{B}_1$ and $\mathbb{B}_2$ occurred during the same period $T$. Nonetheless, we can assume, as we've been assuming, that they at least occur during a nearby period, one in which the methods of production and subsequently the price structure of the economy are relatively unchanging. Thus, ideally we have that $\bar{\Psi}_1$ and $\bar{\Psi}_2$ are identically distributed, so that $E(\log(\bar{\Psi}_1)) - E\log(\bar{\Psi}_2)) = 0$, or more realistically, they are \textit{almost} identically distributed, in which case the difference is very close to $0$. Thus
\[ E(\log(H)) \approx E(\log(C)) \]
with equality if market transactions for the two bundles occurred during the same period $T$. But since $C > 1$ with probability $1$, it follows that $E(\log(C)) > 0$, from which $E(\log(C)) > 0$, from which $E(\log(H))$ is almost certainly also greater than $0$. \par 
Now consider a sequence of $n$ changes over time, given by the commodity bundles $\mathbb{B}_1,\mathbb{B}_2,...,\mathbb{B}_n,\mathbb{B}_{n+1}$. For each transition $\mathbb{B}_i \to \mathbb{B}_{i+1}$, there is a corresponding lcr $H_i$, and by the above discussion it follows that we can reasonably assume that $E(\log(H_i)) > 0$ for each $i$. Define 
\[ H^*_n = \frac{\bar{\Lambda}_1}{\bar{\Lambda}_{n+1}} \]
I.e. we are interested in the \textit{overall} change in labour content of the non-labour inputs during the entire history of the firm. Note that 
\[ H^*_n = \frac{\bar{\Lambda}_1}{\bar{\Lambda}_{n+1}} = \frac{\bar{\Lambda}_1}{\bar{\Lambda}_2}\frac{\bar{\Lambda}_2}{\bar{\Lambda}_3}\ldots \frac{\bar{\Lambda}_n}{\bar{\Lambda}_{n+1}} = \prod_{i=1}^n \frac{\bar{\Lambda}_i}{\bar{\Lambda}_{i+1}} = \prod_{i=1}^n H_i \]
Thus by properties of logarithms
\[ \log(H^*_n) = \sum_{i=1}^n \log(H_i) \]
and subsequently by linearity of expectation,
\[ E(\log(H^*_n)) = \sum_{i=1}^n E(\log(H_i)) \]
From this it becomes clear that 
\[ \lim_{n \to \infty}P(\log(H^*_n) > 0) = 1 \]
The only assumption required to conclude this from our expression is to have it be the case that the expectations of the $\log(H_i)$ approach $0$ at a very high speed, which seems absurd. F and J justify this scenario by comparison to why the cumulative gains of a casino are nearly guaranteed to be positive if the expected gains of each of the games in the casino are positive (i.e. if each game is at least a little bit unfair). This is intuitively trivial: if one expects to lose money on any game they play in the casino, however little and whatever the distributions actually are, then one would expect that the more games one plays, the likelier it is that the luck of the player eventually runs out, and they end up in the red. That's exactly what is happening here. They claim the result follows from the law of large numbers, which it does if one assumes that the distributions of the $H_i$'s are all identical and independent, but neither of this assumptions should be necessary. It would have been nice if F and J provided a set of minimal conditions for this to be the case, but alas, they don't, and I don't have the time to come up with one, for a claim that is so obviously true. \par 
To summarize, what we have shown is that, under the sole assumption that all changes to the non-labour inputs to production of a firm are cost-saving, repeated use of strategy (4) will guarantee an eventual reduction in the labour content of the bundles of commodities being swapped. Across the entire economy then, as firms in all sectors update and innovate, that the labour content of all commodities must decrease over time, with probability approaching one as time approaches infinity. \par 
The next question to ask is how quickly this process occurs. First, we can note with no further assumptions that an even marginally high probability of the $\log(H)$ terms in general being positive will have an exponential effect on the growth rate of $H^*_n$ with $n$, since the dependence of $H*_n$ to the individual $H$'s is multiplicative. With nearly no assumptions at all, we can show it to be reasonable to assume that $P(\log(H) > 0)$ is at least $\frac{1}{2}$. To see this, note that since $C$ is always assumed greater than $1$, $\log(C)$ is always greater than $0$, and so
\begin{align}
	P(\log(H) > 0) &= P(\log(C)+\log(\bar{\Psi}_1) - \log(\bar{\Psi}_2) > 0) \\
				   &= P(\log(C) + \log(\bar{\Psi}_1) > \log(\bar{\Psi}_2)) \\
				   &\geq P(\log(\bar{\Psi}_1) > \log(\bar{\Psi}_2)) \\
				   &\approx \frac{1}{2}
\end{align} 
Where the final approximation follows from symmetry by the fact that $\Psi_1$ and $\Psi_2$ are approximately identically distributed. This fact alone is enough to conclude that the reduction in labour content over time will not be sluggish. However, we can proceed further still than this, by conducting an analysis of the standard deviation of our $\log(H)$ random variables. The speed at which labour content falls depends, among other things, on how concentrated around the mean this distribution is. We know that $E(\log(H))>0$. If the standard deviation is small, then $\log(H)$ will have a very high probability of being very close to the mean, and thus a very high probability of being positive. So far, we haven't made any assumptions at all about the actual distributions of any of these random variables, aside from that $P(C > 0) = 1$ whenever the two periods determining $C$ involve a change in the bundle of non-labour inputs. Nor have we made any assumptions of the relations to each other. In order to make analyze the speed at which this might be expected to occur, we will now highlight some reasonable assumptions which greatly influence this. The first such assumption is that $C$ is generally independent of $\bar{Psi}_1$ and $\bar{\Psi}_2$. This one is fairly innocent. Recall the capitalist is strictly concerned with reducing prices, and are not directly looking at specific prices at all. Thus the specific prices should from the psychological motivations of the capitalists generally be considered independent. Our second assumption is that the correlation random variables $\log(\bar{\Psi}_1)$ and $\log(\bar{\Psi}_2)$, is non-negative. This is because when switching from one commodity bundle to another, it will rarely if ever be the case that there is \textit{nothing} in common between the two bundles. Usually, almost always, there will be commodities in common, and this would create a positive correlation. If $\log(\bar{\Psi}_1)$ and $\log(\bar{\Psi}_2)$ are nonnegatively correlated, then of course $\log(bar{\Psi}_1)$ and $\log(\bar{\Psi}_2)$ are non-positively correlated, and so  
\begin{align}
	Var(\log(H)) &= Var(\log(C)+\log(\bar{\Psi}_1)-\log(\bar{\Psi}_2)) \\
			&= Var(\log(C))+Var(\log(\bar{\Psi}_1)) + Var(\log(\bar{\Psi}_2)) - 2Cor(\log(\bar{\Psi}_1),-\log(\bar{\Psi}_2)\\
			&\leq Var(\log(C))+Var(\log(\bar{\Psi}_1)) + Var(\log(\bar{\Psi}_2))
\end{align}
Since $\bar{\Psi}_1$ and $\bar{\Psi}_2$ are approximately identically distributed, the two variance terms are the same. Let's consider what this is exactly, and to do this we must look back at our formula for what the aggregate specific price is in the first place. Namely, it is a weighted sum of $m$ specific price random variables, which can be reasonably assumed independent of each other, and also reasonably assumed normal. $m$ itself is likely quite large, typically in the hundreds if not thousands. By both properties of sums of normal random variables then, or by the central limit theorem, we can say with, if anything, more confidence than we did before about the ordinary specific price distribution, that this thing is normal, with mean $E(\Psi)$ and standard deviation $\frac{SD(\Psi)}{\bar{m}}$, where $bar{m}$ is the average number of non-zero terms (i.e. $E(M)$, and still quite large). It was estimated earlier that $SD(\Psi)$ was less than a third, and so this standard deviation, proportional to how diverse the labor inputs are to capitalist firms, is very, \textit{very} small. \par 
The upshot is that it is very likely that $Var(\log(\bar{\Psi}_1)) + Var(\log(\bar{\Psi}_2))$ is quite small, so that \textit{the distribution of $\log(H)$ is just a bit narrower than that of $\log(C)$, likely extremely similar}. Now recall that we observed earlier that $E(\log(H)) \approx E(\log(C))$, and also have that since $P(C > 1) = 1$, that $P(\log(C) > 0) = 1$, and that $E(\log(C)) > 0$. From all of this it follows that $P(\log(H) > 0) \approx 1$. A \textit{single} change in non-labour inputs is extremely likely to drop the labour content! 
\subsection{Accumulation and the Rate of Profit}
Farjoun and Machover make the important point that effective modeling of accumulation cannot be measured by simply measuring the total raw amounts of goods over time, since commodity types are constantly being abandoned in favor of new commodity types. The only way to effectively track this is by measuring the contents of some chosen universal, in our case of course this being labor content. \par 
Recall the general rate of profit, which we saw to also be the expected rate of profit, was given by
\[  r_g = \frac{\pi(\mathbb{P}) - \pi(\mathbb{I}) - \pi(\mathbb{V})}{\pi(\mathbb{K})} \approx \approx \frac{\lambda(\mathbb{P}) - \lambda(\mathbb{I}) - \lambda(\mathbb{V})}{\lambda(\mathbb{K})}  \]
Recall also that 
\[ \lambda(\mathbb{P}) \approx \lambda(\mathbb{C}) \approx n_l \]
Simplifying the notation by letting $\lambda(\mathbb{V}) = \nu$ and $\lambda(\mathbb{K}) = \kappa$, we have
\[  r_g \approx \frac{n_l - \nu}{\kappa} \]
Here $n_l - \kappa$ is clearly very closely analogous to Marx's $S$, the total surplus value generated by the economy, or said differently, the total labor which is alienated from the workers themselves. Similarly $\kappa$ is analogous to $C+V$, in Marx's model, and so we can see that $r_g$ itself is analogous in Marx's model to $\frac{S}{C+V}$, which is exactly what he defined as the rate of profit. From this we can see that the general rate of profit is directly proportional to the "surplus value generated" $S$ and inversely proportional to the total capital. \par 
An important observation can be made here. If the economy is "closed", that is, then the sole source of growth is reinvestment of the surplus produced. If the entire surplus were reinvested, it would add $n_l+\nu$ units of labor content per period $T$. Thus we have, in continuous terms,
\[ \frac{d\kappa}{dt} < n_l-\nu \]
where the strict inequality follows from two different observations. First, it is impossible to truly reinvest all surplus into production, since if this were done the capitalist class would quickly starve to death. Second and more importantly, it is important to clarify something which should already have been clarified: the labor content of a commodity during a particular period is to be measured in terms of the \textit{current} technical coefficients of that period. If a commodity is produced in period $T_1$, and used as capital during $T_2$, and it happens that the labor content of this commodity type has shrunk due to the law of decreasing labor content between these periods, then the labor content of the commodity will have shrunk. This drop means that while the increase in $\kappa$ over time is bounded by $n_l-\nu$, there is also a reverse force pulling this value down all of the time. \par 
Farjoun and Machover view $n_l-\nu$ as the source of increasing capital flows, and the law of decreasing labor content as a sink. They identify two more sinks: destruction of capital stock due to war, and destruction of capital stock during times of economic crisis. \par 
With these observations in hand, let's review Marx's own explanation of the tendency of the rate of profit to fall, but within the vocabulary developed here. Marx defined his rate of exploitation as $e = \frac{S}{V}$, his rate of profit as $r = \frac{S}{C+V}$, and his organic composition of capital as $o =\frac{C}{V}$. He argued that the organic composition would tend to rise over time: due to labor content of commodities tending to decrease over time, subsistence wages would be allowed to fall, producing a drop in $V$, and that this reduction in real wages would be accompanied by an increase in the reliance on heavy machinery, producing a rise in constant capital $C$. A mathematical way to present the consequences of this would be to first re-express the rate of profit in terms of the rate of exploitation and the organic composition:
\[ r = \frac{S}{C+V}\times \frac{\frac{1}{V}}{\frac{1}{V}} = \frac{e}{1+o} \]
From here it is plain to see that if $o$ is to continually increase, this immediately produces a decrease in the rate of profit, unless accompanied by a simultaneous increase of exploitation. Thus, the capitalist class will continually operate under the incentive of increasing exploitation, since this will be the only way to stop their profit rates from falling. \par 
Now, let's put this in our own terms. In place of his rate of exploitation, we have our global rate $e_g = \frac{n_l-\nu}{\nu}$. Our own general rate of profit can be expressed in terms of this rate:
\[ r_g \approx \frac{n_l-\nu}{\kappa} = \frac{e_g\nu}{\kappa} = \frac{\nu}{\kappa}e_g \]
This equation is in full agreement with Marx's model, since $\frac{\nu}{\kappa}$ is analogous to $\frac{V}{C+V}$, which when inverted becomes $1+\frac{C}{V}$, which is exactly what we find in the denominator or $r$. Thus this equation is exactly analogous to Marx's in the way we would expect - we are simply operationalizing it in our own terms. Towards continuing this process, let us define 
\[ q_g = \frac{\kappa}{\nu} \equiv 1+o \]
The difficulty of this argument is that in order to actually view this as a tendency for the rate of profit to fall, we must produce an argument for why the rate of exploitation $e_g$ will generally fail to rise fast enough to counteract the rise of $q_g$. Farjoun and Machover go in a different direction from this, and their results are quite interesting. First, they solve for $\nu$ in the equation for $e_g$:
\begin{align}
	e_g &= \frac{n_l-\nu}{\nu} = \frac{n_l}{\nu}-1  \\
	&\implies e_g+1 = \frac{n_l}{\nu} \implies \nu = \frac{n_l}{1+e_g} 
\end{align}
Then 
\begin{align}
	r_g = \frac{\nu}{\kappa}e_g = \frac{n_l}{(1+e_g)\kappa}e_g = \left(\frac{n_l}{\kappa}\right) \left(\frac{e_g}{1+e_g} \right)
\end{align}
This is a fascinating equation, completely consistent with Marx's model but obscured by the fact that within that model, there wasn't a simple variable corresponding to the total labor done by society $n_l$. In our model, $n_l$ is the number of units of labor power employed during the period $T$, while $\kappa$ is the labor content of the total capital employed by society during the period $T$. Now, as a function of $e_g$, $\frac{e_g}{1+e_g}$ is strictly increasing, and thus the rate of profit here is still proportional to $e_g$. But this proportionality is no longer direct, and in fact, reveals the inadequacy of the rate of exploitation as a correcting factor. This function of $e_g$ is in fact \textit{bounded}, converging to $1$ as $e_g \to \infty$. Thus, if the first term $\frac{n_l}{\kappa}$ can and tends to decrease without bound (a claim we are about to consider), an accompanying increase of the global rate of exploitation $e_g$ can only temporarily prevent the global rate of profit from decreasing over time; it is powerless to stop it in the long term. \par 
Let's compare the term $q_g' = \frac{n_l}{\kappa}$ with the term $q_g$. What we currently have is two equations for the general rate of profit:
\[ r_g \approx e_g\frac{1}{q_g} \approx \left(\frac{e_g}{1+e_g}\right)\frac{1}{q_g'} \]
Clearly the two terms $q_g$ and $q_g'$ are similar, and in fact the general rate of profit is inversely related to both. For a fixed period $T$, $q_g = \frac{\kappa}{\nu}$ is the ratio of the total labor content of all capital used in production to the total labor content of real wages. $q_g'$ is the ratio of the total capital used in production to \textit{the total labor employed during the period} - or, more simply, the ratio of total capital employed per worker, measured in labor time. Marx seems to conflate these two terms, and assume that $q_g$ increasing necessarily implies an increase in $q_g'$. For these terms to be the same, the workers would have to be receiving \textit{all} of their labor back in the form of wages! So they are absolutely different quantities, but should be directly relatable to each other in terms of the global rate of exploitation $e_g$, since, as we have just noted, equality would imply that $e_g = 0$. \par 
Since $q_g = \frac{\kappa}{\nu}$, and $\nu = \frac{n_l}{1+e_g}$, substituting gives $q_g = \frac{\kappa(1+e_g)}{n_l}$. Thus
\[ q_g = (1+e_g)q_g' \]
This is totally consistent with what we just observed: $q_g$ is at all times some percentage bigger than $q_g$, the difference completely attributable to the amount of labor being alienated from workers to the capitalist class. Nonetheless we see that in some sense, Marx was justified in conflating his organic composition $q_g$ with this quantity - they are directly proportional. An increasing organic composition implies an increasing of the quantity $\frac{\kappa}{n_l}$, provided that the rate of exploitation stays constant. However, if the rate of exploitation were to simultaneously increase, this conflation would be in error. This is why the two equations for the general rate of profit are non-contradictory. In a sense, $q_g'$ is the technical ratio of capital per laborer. It is the organic composition as Marx discussed it, completely divorced from whatever exploitation is happening. \par
Nonetheless, Farjoun and Machover connect back to their earlier empirical evidence, namely the fact that from what they observe, $e_g$ is constant across developed capitalist economies and unchanging with time. Thus, in the context of our own world, it would seem that $q_g$ and $q_g'$ \textit{are in fact directly proportional}. Recall now the random variable defined over firm space, $Z = \frac{R}{K}$ - the proportion of wages relative to total capital of a firm. We saw then that $E(Z) = \frac{\pi(\mathbb{V})}{\pi(\mathbb{K})}$. Recall however that for large aggregates of commodities $\mathbb{B}$, $\pi(\mathbb{B}) \approx E(\Psi)\lambda(\mathbb{B})$. Since $\mathbb{V}$ and $\mathbb{K}$ are indeed enormous aggregates of commodities, we then have that
\[ E(Z) \approx \frac{\lambda(\mathbb{V})}{\lambda(\mathbb{K})} = \frac{\nu}{\kappa} = q_g \]
We mentioned earlier that the random variable $\frac{1}{Z}$ was analogous to Marx's organic composition of capital (plus $1$), and we now have, in a way, justified that claim, in the sense that $\frac{1}{E(Z)}$ is exactly (or rather, approximately) that. If we let $Q = \frac{1}{Z}$, we can then see it as the random variable measuring the organic composition of a firm. Note however that $Q$ is measured in price categories, not labor categories, and with the inversion of the $Z$ this produces an artifact in the sense that it is not the case that $E(Q) \neq \frac{1}{E(Z)}$. Since $Z$ was conjectured quite reasonably to have a gamma distribution, it follows that $Q$ has an \textit{inverse gamma distribution}. If the parameters of the gamma distribution are $\alpha$ and $\beta$, then these are also the parameters of the inverse gamma, though their role in the pdf and subsequently the mean and variance are different. In particular, the mean of an inverse gamma is $\frac{\beta}{\alpha-1}$, which is precisely $1$ over the \textit{mode} of the distribution of $Z$. However, since the expected value of a gamma distribution is $\frac{\alpha}{\beta}$, we have that
\[ q_g \approx \frac{1}{E(Z)} = \frac{\beta}{\alpha} \]
and that
\[E(Q) = \frac{\beta}{\alpha-1} \]
It is clear that despite $E(Q)$ not technically equaling $q_g$, it will still be very similar, at all times just a little big larger. (Which makes perfect sense, since relating the capital to labor in terms of prices produces distortion attributable to differences in markup due to exploitation, which due to exploitation deflates the wages relative to the labor actually done. Thus the denominator of the former will always be a bit smaller than what it should be, an artifact produced by exploitation.)  
\par In any case, the equation $r_g = \frac{1}{q_g'}\left(\frac{e_g}{1+e_g}\right)$ makes for a much cleaner argument in terms of what Marx was trying to show, and is consistent with Marx's thinking himself (although this is slightly attributable to a mistake). Since he conflated $q_g'$ with $q_g$, the argument he gave that $q_g$ grows without bound applies by his own reasoning to $q_g'$ as well, and as we have seen this is not entirely without merit. It makes for a cleaner overall argument that the profit rate will actually \textit{fall} over time, since it makes clear that $e_g$ cannot effectively compensate. However, Farjoun and Machover set up this argument in some sense just to knock it down. As is well known the empirical evidence for a falling rate of profit is very mixed and unclear, and Farjoun and Machover in fact argue that not only does the rate of profit not necessarily have a tendency to fall over time, but that it is in fact bounded. \par 
They first point out that $e_g$, as approximated to the more easily measurable $e_0$ (which recall is simply the ratio of profits to wages in a country), is more or less constant in developed countries, the same in all of them, and barely changes at all over time. Since it barely changes, it cannot be seen to influence the rate of profit much at all; $r_g$ depends entirely on either $q_g$ or $q_g'$. They note then that since there has been no observable downward trend over the last several generations, it could not be the case that $q_g$ or $q_g'$ have changed significantly. I personally find this argument not at all convincing, since if one is measuring profits in a particular country, they are likely only observing the wages of workers in that country, meaning that wages paid to employees of the firm that are working in \textit{other} countries are being ignored. Thus they seem to be missing the inflationary effect of, simply put, imperialism. \par 
Nonetheless, they go on to make an independent argument that, in lieu of simply the rate of profit being a non-uniform random variable, that it will not only not fall over a long period of time but in fact will be bounded. Their argument is worth understanding. It hinges on the simple observation that \textit{if the rate of profit is a random variable, then the health of the economy as a whole depends not on the average rate of profit, but on the distribution as a whole.} To illustrate this, they point to a table of different gamma distributions with different means but each with standard deviations which are half of that. For example, if $X \sim Gamma(4,20)$, then $E(X) = .2$, and $SD(X) = .1$, and if $Y \sim Gamma(4,40)$, then $E(Y) = .1$ and $SD(X) = .05$, so these represents a simple case of an economy which goes from a global rate of profit of $.2$ to a global rate of $.1$, a ten percent drop in profitability. Since a ten percent profit rate is still plenty good, an economist viewing the rate of profit strictly by its average wouldn't see this drop as particularly catastrophic, but the fact is that when at $.2$, $1.69$ percent of firms are operating at below a $5$ percent profit rate, but at $.1$, that number jumps to $14.3$ percent. That's more than $10$ percent of firms the economy suddenly on the brink of being unable to cover their debts. When the average rate of profit drops to $.05$ percent, the number below a $5$ percent profit rate jumps to $56.7$ percent. Their point is that the economy will be perceived to be in crisis \textit{well before} the average rate of profit is at a point where most economists will acknowledge that a crisis has arrived. The probabilistic approach also reveals how even in times of deep crisis, many firms in the economy will remain doing quite well for themselves. Farjoun and Machover, after making this observation, illustrate some of the reactive mechanisms which will begin occurring well before a crisis of the sort that an economist would start their thoughts at.\par 
By their earlier empirical observations, they are assuming that regardless of whatever else is going on, $e_g$ remains constant, and focus on inspecting $r_g$ through the variable $\frac{k}{n_l}$, which is inversely proportional to the rate of profit. First, as the rate of profit falls, less profitable firms go out of business, and their capital is written off. This reduces $k$. The crisis also stimulates competition, incentivizing well off firms to quickly adapt the innovations developed between the last crisis and the current one, increasing productivity of labor and thus the ratio of capital to labor. Both of these factors cause a decrease of $\frac{k}{n_l}$, and subsequently an increase of $r_g$. \par 
Conversely when the boom times have been sufficiently initiated, profits are flowing more rapidly than usual, and there is less incentive to innovate. Capitalists are reinvesting their capital in increasing amounts, but they are doing so in a way that merely scales production up without changing the method at all. This invokes the law of falling marginal productivity. For example, in the business of coal mining, less rich mines are brought into operation which require more labor to produce the same amount of coal. Thus the labor content of each ton of coal tends to rise. This happens across the entire economy, causing $\frac{k}{n_l}$ to increase, beginning the process of $r_g$ reversing course and eventually beginning to fall again. \par 
Farjoun and Machover clearly feel that, in lieu of their observation that the capitalist economy will be perceived by its inhabitants as in crisis well before the profit rate actually falls to a dramatic degree, this extremely standard description of the industrial cycle, and in particular the mechanisms which bounce $r_g$ back up again, trigger earlier than one would expect. This only serves to reinforce the more heavy lifting parts of their argument which allow them to conclude that the rate of profit is \textit{bounded}. Here is how the conclusion is reached: $e_g$ is unchanging, and thus $r_g$ depends entirely on $\frac{k}{n_l}$, which oscillates as described above by the opposing tendencies of innovation and quantitative expansion. Thus $r_g$ is bounded if and only if $\frac{k}{n_l}$ is bounded. Their argument for this was made earlier: empirical evidence suggesting that the rate of profit has not fallen, along with empirical evidence suggesting that $e_g$ does not change much at all, leads one to conclude that $q_g'$ remains unchanging as well. Thus the rate of profit is bounded. Casting aside the strange circularity of this argument, which effectively says that "the rate of profit is theoretically bounded because we have observed that it is bounded", I don't find it convincing at all, because the idea that $\frac{k}{n_l}$ would remain unchanging for the duration of capitalism's history appears as absurd to me. Not only that, I would wager any amount of money that if one were to account for the child laborers employed in the DRC mining minerals for the production of silicone chips manufactured in Taiwan, then the $e_g$ of Taiwan would be clearly observed to have fallen significantly after that exporting of capital. 

