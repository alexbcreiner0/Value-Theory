\section{Exploitation, Price, and Profit}
I've now laid out the foundational plumbing for the model: I've defined the three spaces, the probability measures on those spaces, the variables (both random and physical) central to these spaces, and documented a set of minimal assumptions which, once made, link them together. Of special significance of course was the linking of specific price $\Psi$ with the rate of profit $R$. Next I would like to turn to what F and J have to say about the actual shapes and parameters of these distributions, and along the way arrive at three different definitions of something which would be analogous to "the general rate of exploitation". \par 
The discussion will be centered around the specific price, $\Psi = \frac{\Pi}{\Lambda}$. Consider the $\auw$ price of the $i^{th}$ transaction $\Pi(i)$. Now, it's tautological to decompose this price into three components, through two stages of decomposition. First, there is the decomposition into cost of production plus profit. Then, the price of production can itself be broken into wages and not wages. Thus we can write for any commodity $C$ which is sold for a price $\pi(C)$, that
\[ \pi(C) = \pi' + v' + s' \]
where $s'$ is the profit (which could be negative), $v'$ is the wages paid (which could be obtained concretely in a variety of wages; there is a discussion which must be had of how to distribute the the total revenue of a firm throughout the products produced and sold by that firm during the period $T$, but for now we accept that such a thing can be done in an objective manner), and $\pi'$ is the amount paid on the materials and production - what Marx would call the constant capital. \par 
If we consider $\pi'$ in relation to $\pi$, it becomes extremely tempting to believe that in general $\pi'$ must be strictly less than $\pi$. This cannot be said in general, but it is extremely likely, approaching certainty. $v'$ is of course non-negative, and can only be $0$ if there exists a commodity which requires no living labour to construct from it's constituent parts, something which can be imagined for land, but very little else. $s' < 0$ would mean the price is lower than the cost of production, which if happening widely would imply a negative profit rate for a significant number of firms. In order for it to not be the case that $\pi' < \pi$, it would need to be the case that the firm sells their commodity at a loss, one which \textit{exceeds} the wages used in it's production. Farjoun and Machover assume for descriptive purposes that this is the case for a commodity in order to demonstrate a deeper identity which can be derived from such an assumption. $\pi'$ itself can be split into pieces, namely the different purchases that the producing firm made from other firms which constitute $\pi'$. Call these $\pi^{''}_1,\pi^{''}_2,...,\pi^{''}_{m}$. For any one of these, say $\pi^{''}_i$, we can apply the same argument again, decomposing into wages $v^{''}_i$, profits $s^{''}_i$, and raw materials $\pi^{'''}_i$, and this third term can itself be broken down into a finite set of subcomponents, and so on. Now, the assumption that$\pi'$ is always strictly less than $\pi$ for \textit{ANY} commodity is a very strong one, and F and J seem concerned with making it seemingly in the hopes of arguing that the sequence of all such $\pi$ terms must go to zero, since it is at that point an essentially monotonically decreasing function. However, the monotonicity here remains tricky even here, since what we have is not actually a sequence, but a directed set, and not a sequence of prices but rather in topological terms a \textit{net}. Thus, they have to make the rather extreme assumption that there is exactly one firm producing all of the raw materials for a given commodity. They do all of this in the hopes of writing $\pi$ itself as a complex sum of wages and surpluses, eventually eliminating entirely the raw materials term. But even with all of their assumptions which guarantee that the sequence of $\pi$ terms approaches $0$, this is of course insufficient to show that the sum of all such terms converges. \par 
This is all a big mess, which Farjoun and Machover plainly admit. What I propose is something simpler. Define, for a given commodity $c_i$ in market space, a set of $2n_f$ many random variables $V_1^c, V_2^c,...,V_{n_f}^c,S_1^c,S_2^c,...,S_{n_f}^c$, where $V_j^c(c_i)$ to be the total money wages paid by the $i^{th}$ firm, directly or indirectly, in the production of $c_i$, and $S_j(c_i)$ is the money profit made by the $i^{th}$ firm in the sale of the product of those wages. These numbers are extremely complex: when we consider the contribution of the $j^{th}$ firm to the commodity $c_i$, we must remember to include \textit{every} contribution they made to \textit{every} firm which \textit{eventually} contributed themselves to $c_i$. For instance, a firm producing steel might sell steel directly to a firm producing cars, but they might also sell steel to another firm producing tires which it eventually sells to the same firm producing cars. In this case we must add the wages paid for \textit{both} the steel directly used, and the wages paid for the steel used indirectly in the production of the tires. I think it is clear that these numbers exist in an objective sense, despite being nearly impossible to calculate perfectly, and despite being very complex can be very effectively approximated. \par 
So, with all of that said, we have, for any commodity in firm space, a slew of new random variables, two for every element of firm space. We can more importantly define, for any commodity $c_i$ the \textit{sum} of the $V$'s and the \textit{sum} of the $S$'s:
\[ V^c(c_i) = \sum_{j=1}^{n_f} V_j^c(c_i) \]
\[ S^c(c_i) = \sum_{j=1}^{n_f} S_j^c(c_i) \]
Where the $c$ superscripts serve to distinguish these random variables to the $V$ and $S$ random variables previously described over firm space. The claim that F and J are concerned with making is that the following is true in general for all commodities:
\begin{align}
	\Pi(c_i) &= \sum_{j=1}^{n_f} V^c_j(c_i) + S^c_j(c_j) \\
			&= \sum_{j=1}^{n_f} V^c(c_i)+S^c(c_j)
\end{align}
Or, more simply, in general, 
\begin{align}
	\Pi = V^c + S^c
\end{align}
To state this as an identity considering the mess of a conversation earlier might seem odious, so let's think about it. Take the steel sold to a firm in production of a car as an example. This steel has a variety of components associated with it, and we can trace it backwards to smaller and smaller contributions. Suppose in tracing back the steel, we eventually land at a firm which chops trees and produces lumber, that this firm pays nothing for use of the trees, and that they for simplicity also produce their own equipment used to chop the trees. That is to say, this is a firm which pays nothing for the raw materials of their own production. Of course, there is a markup at this point; the firm profits, meaning that they markup the price from what wages were used to produce the lumber, and these profits can easily be seen as a markup based on monopoly ownership of the lumber and materials. But if we were to relativize the theory to some other commodity, say oil, then this same decomposition would be \textit{impossible}, since assuming it ever terminates in the sense that it does for the lumber firm would imply that the firm owns the labour power used, which would be slavery. It appears to me that F and J's mathematical attempts to justify this identity made things more confusing than they needed to be, and that what we are saying here is not only valid, but perhaps an essential piece of evidence towards the logical \textit{necessity} of viewing a capitalist economy in terms of labour power. \par 
With all of that said, we can now turn to specific price. Note the identity we now have for it:
\[ \Psi = \frac{V^c}{\Lambda} + \frac{S^c}{\Lambda} \]
This by itself says something significant, if we note that
\begin{align}
	 V^c + S^c &= \sum_{i=1}^{n_f} V^c_i + \sum_{i=1}^{n_f} S^c_i \\
	 		   &= \sum_{i=1}^{n_f} (V^c_i + S^c_i)
\end{align}
And so 
\[ \Psi = \frac{V^c_1+S^c_1}{\Lambda} + \frac{V^c_2+S^c_2}{\Lambda} + \ldots + \frac{V^c_{n_f}+S^c_{n_f}}{\Lambda} \]
So $\Psi$ is the sum of an extremely large number of at the very least similarly distributed random variables which are weakly correlated and low in variance, which by the central limit theorem heavily insinuates that the overall sum has a is normally distributed. They argue that due to the integrated nature of modern capitalism, hundreds if not thousands of these terms are necessarily nonzero, and, since "even the biggest firms are quite small relative to the economy as a whole", most of these terms are small relative to each other. In any case, this is a claim which requires empirical testing to confirm, but considering how loose the conditions are in order for some version of CLT to apply, it seems fairly apparent in lieu of this decomposition that the distribution is typically mostly normal. \par 
More than this even, this new expression for $\Psi$ allows a closer inspection of the expected value. Note
\[ E(\Psi) = E\left(\frac{V^c}{\Lambda}\right)+E\left(\frac{S^c}{\Lambda}\right) \]
Considering the first of these two terms, we see that
\begin{align}
	E\left(\frac{V^c}{\Lambda}\right) &= \sum_{i=1}^{n_m} \frac{V^c(c_i)}{\Lambda(c_i)}P(c_i) \\
		&= \sum_{i=1}^{n_m} \frac{V^c(c_i)}{\Lambda(c_i)}\frac{\Lambda(c_i)}{\sum_{j=1}^{n_m}\Lambda(c_j)} \\
		&= \frac{\sum_{i=1}^{n_m}V^c(c_i)}{\sum_{j=1}^{n_m}\Lambda(c_j)} \\
\end{align}
The numerator here is the total amount of money paid in wages for the production of all of the commodities sold in the period $T$. Now, again we must be careful to remember that the commodities bought/sold during a period of time is not the same as the commodities produced during that same period. Thus the number here in the numerator here is \textit{not} simply the total amount paid in wages, which we denoted $S$ earlier when defining the $\auw$ unit and the wages random variable $W$. However, it will generally be quite close to $S$, given a few assumptions. In fact, these assumptions are very similar to two of the assumptions  made earlier. In order to safely assume that $S$ is a good approximation of the numerator here, we must have that:
\begin{itemize}
	\item[(1)] The vast majority of commodities bought and sold during the period $T$ are commodities which are either produced during the current period or during one of the nearby periods
	\item[(2)] The real wages paid (i.e. the quantity $S$) changes relatively slowly between periods. 
\end{itemize}
Turning to the denominator, what we have is the total labour content of the entire collection of commodities bought/sold in the period $T$, i.e. $\lambda(\mathbb{C})$ (where recall $\mathbb{C}$ is the physical bundle of all commodities in market space). Again, we are faced with the question of whether or not this amount of labour time is the same as the total labour done \textit{during} the period $T$, which is of course $n_l$. Assumption (1) above is enough to assume this to be the case, though with likely a bit less accuracy. What we are left with then is that, under assumptions (1) and (2), 
\[ E\left(\frac{V^c}{\Lambda}\right) \approx \frac{S}{n_l} = 1\auw = E(W) \]
So this term is approximately $1$, with proximity to one dependent on the extent to which capitalism perfectly "reproduces itself" between periods. Depending on the choice of $T$, it appears to me that this approximation could be quite a bit aways from $1$, but it is nonetheless an interesting observation. \par 
Next, we turn to the second term $E\left(\frac{S^c}{\Lambda}\right)$. An identical sequence of steps gives us that
\[ E\left(\frac{S^c}{\Lambda}\right) = \frac{\sum_{i=1}^{n_m} S^c(c_i)}{\sum_{j=1}^{n_m} \Lambda(c_j)} \]
This term is similar enough to the other one that they can be related to each other. Define 
\[ e^* = \frac{\sum_{i=1}^{n_m} S^c(c_i)}{\sum_{i=1}^{n_m}V^c(c_i)} \]
Then 
\begin{align}
	E\left(\frac{S^c}{\Lambda} \right) &= \frac{\sum_{i=1}^{n_m} S^c(c_i)}{\sum_{i=1}^{n_m} \Lambda(c_i)}\frac{\sum_{i=1}^{n_m} V^c(c_i)}{\sum_{i=1}^{n_m} V^c(c_i)} \\
	&= \frac{\sum_{i=1}^{n_m} S^c(c_i)}{\sum_{i=1}^{n_m} V^c(c_i)}\frac{\sum_{i=1}^{n_m} V^c(c_i)}{\sum_{i=1}^{n_m} \Lambda(c_i)} \\
	&= e^*E\left( \frac{V^c}{\Lambda} \right)
\end{align} 
And so we have, with only the assumptions implicit to the model itself and the claim that price can be broken up into wages and markup, the identity
\begin{align}
	E(\Psi) =  E\left(\frac{V^c}{\Lambda}\right)+E\left(\frac{S^c}{\Lambda}\right) &= E \left( \frac{V^c}{\Lambda} \right) + e^* E \left( \frac{V^c}{\Lambda} \right) \\
	&= (1+e^*)E\left(\frac{V^c}{\Lambda}\right)
\end{align} 
And if we include our minimal equilibrium assumptions about a capitalist economy which result in $E\left(\frac{V^c}{\Lambda}\right) \approx E(W) = 1$, then we have
\[ E(\Psi) \approx 1+e^* \]
Remember that $E(\Psi)$ is this model's analog of $\psi_0$, Marx's constant of proportionality for the claim that prices are proportional to labour times. Thus $1+e^*$ here denotes \textit{a general markup on commodity process as one hundred plus $e^*$ percent of the labour time embodied in the commodity}. What is $e^*$, exactly? I will call this the \textbf{market rate of exploitation}, do distinguish it from the two similar numbers I'll define shortly, and highlight that it is a number defined purely within market space, rather than firm or labour-power space. What this number is, at face value, is the ratio of total markup of all commodities produced in a period of time $T$ versus total wages paid in that production. Purely from inspection of the number then, despite what I've decided to call it, this doesn't seem to highlight with any great that profits from the sale of commodities actually come from the expropriation of surplus labour from the worker. After all, there's no telling where the markup in prices came from, or why these markups were accepted by consumers. \par 
However, we can define a similar number from our other two spaces. First, consider labour-power space. The total labour done in the period $T$ is of course $n_l$. We also have the bundle of commodities $\mathbb{V}$ which constitute the real wages paid to these workers, i.e. the bundle of goods consumed by workers during this period. Then the difference $n_f - \lambda(\mathbb{V})$ \textit{directly} has an exploitative character: it is exactly how many units of labour power had no material benefit for the workers, labour which they offered up as tribute to the capitalist economy. We can then define a much more provocative \textbf{global rate of exploitation}
\[ e_g = \frac{n_l - \lambda(\mathbb{V})}{\lambda(\mathbb{V})} \]
A few caveats are in order before messing around too much with this number. First, we should note that there is no guaruntee that workers aren't purchasing their goods exclusively with the wages they earned during the period $T$. This muddies the initial claim that the difference in the numerator is \textit{exactly} the amount of labour offered up as tribute during the period $T$. However, the vast majority of workers in a modern economy spend money they earn during a pay period \textit{during} that period, and save relatively little of it. Thus, the claim, while not being exact, is \textit{far} from outright invalid, and that's without invoking any kind of confusing periodicity argument, which could be done if I felt like giving myself a headache. \par 
Now, turning directly to the matter at hand, the bundle $\mathbb{V}$ is absolutely massive, extremely varied, and very much unbiased. Workers in Marx's day may not have consumed much in the way of industrial equipment in the sense of steel, oil, electricity and so forth, but now they \textit{do}; with few exceptions they consume a bit of everything, and the exceptions that exist only exist at a "top level" (for example, workers don't spend their wages purchasing military drones, but they very likely \textit{do} spend their wages purchasing products made from the same materials which were used in order to create those drones). Thus, we can apply our statistical version of Marx's labour theory of value:
\[ \frac{\pi(\mathbb{V})}{\lambda(\mathbb{V})} \approx E(\Psi) \approx 1+e^* \]
What is $\pi(\mathbb{V})$? It's the wages paid to workers in actual money units. Particularly, units of $\auw$. However, from the definition of this unit, this total wage amount is simply itself $n_l$ (recall that if $S$ is the total wages in dollars paid to workers, then $1\auw = \frac{S}{n_l}$, and so $S$ dollars in $\auw$ is $S \dollars = S \times \frac{n_l}{n_l} \dollars = n_l \times \frac{S}{n_l} \dollars = n_l \auw$.) So $\pi(\mathbb{V}) \approx n_l \auw$ (approximate because, again, workers will save some of their money to spend at a later period, and buy some commodities using money saved from a previous period). Thus,
\begin{align}
	e_g &= \frac{n_l}{\lambda(\mathbb{V})} - 1 \approx \frac{\pi(\mathbb{V})}{\lambda(\mathbb{V})} - 1  \\
		&\approx (1+e^*)-1 = e^* \\
		\implies e^* \approx e_g
\end{align}
Suddenly then, the seemingly innocent markup of prices due to $e^*$ takes on a much more exploitative significance - profits suddenly seem to suddenly depend entirely on the exploitation of labour power. However, we're getting a bit ahead of ourselves. It's worth re-emphasizing here that the number $e^*$ \textit{is uniquely definable in a model of the sort we are working with relativized to labour as the central commodity.} The decomposition of price into wages and surplus would not be obviously valid (to me at least), relativized to oil, unless we were to view capitalist firms as interchangeable with individuals, or at least that every firm is a worker co-op, because otherwise we would have to assume that firms own the labour-power of their production, which would be slavery. \par 
We've linked profits from prices to exploitation of labour, but what about the rate of profit of a firm? Let's turn back now to firm space, and see what we can discern. Here, we have the rate of profit $R$ of a firm, the capital of the firm $K$, the portion of capital in the form of wages $V$, and the proportion of wages to total capital $Z = \frac{V}{K}$. We saw that with the basic assumption that capitalist firms reinvest a portion of profits into production that the capital of a firm grew exponentially with $R$ directly related to the base of that exponential function, and we also saw that assuming a constant "technical composition of capital" that the behavior of $Z$ with time mimicked the behavior of $R$. (We in fact saw that the reciprocal $\frac{1}{Z}$ was nearly identical to what Marx referred to as the organic composition of capital.) Finally, we defined, but did not discuss much, the rate of capital versus labour $X = \frac{R}{Z}$, which we saw was analogous to Marx's famous formula for the rate of exploitation $\frac{S}{V}$. This variable $X$ is obviously the one we need to look at very closely now. In particular, let's look at the number 
\begin{align}
	e_0 &= \frac{E(R)}{E(Z)} \\
	&= \frac{\pi(\mathbb{P}) - \pi(\mathbb{I}) - \pi(\mathbb{V})}{\pi(\mathbb{V})}
\end{align}  
That is, $e_0$, clearly related though not exactly equal to $E\left( \frac{R}{Z} \right) = E(X)$, is precisely the ratio of total profits to total wages in the economy. This is perhaps even more explicitly related to exploitation than $e_g$: the bigger it is, the bigger the tributary products of worker's labour power is leaving the worker's hands and not contributing directly to their well being (outside of government intervention, at least). \par 
Also more direct is the immediately discernable relationship between $e^*$ and $e_0$. It appears at first glance that they are identical: $e^*$ is the total profit of the economy computed in terms of overall markups in prices of commodity sales, and $e_0$ is the total profit of the economy computed in terms of aggregating the overall profits of each firm. How these differ of course is in the same ways that anyone reading this is probably used to at this point: commodities bought and sold don't identically correlate with those produced in a fixed period $T$, and profit rates and labour contents change between periods. Thus, in order to assume that $e_0 \approx e^*$, we must assume:
\begin{itemize}
	\item[(1)] The commodities bought and sold during the period $T$ are mostly the same as those produced in the period $T$, or at least produced in a preceding period close to the present one.
	\item[(2)] The distribution of $X$ changes relatively slowly over time, i.e. between periods (more directly, this is equivalent to saying that $R$ and $Z$ change relatively slowly with time).
\end{itemize}
Thus we have three numbers measuring exploitation of labour in a capitalist economy, all roughly equal with one another with fairly minimal and loose conditions of equilibrium/normality in the economy. These numbers link the three "viewpoints" of the capitalist economy, that of the market, that of the firm, and that of labour, together into a coherent model which seems to not so much as imply a labour determined system but \textit{command} it, \textit{necessitate} it, and that was despite my attempt to remain as unbiased towards labour as possible, at every possible moment noting the extent to which the choice of labour as the central relative commodity is arbitrary. \par 
\subsection{Estimating the Shape of the Distributions}
Before finishing this section, we'll note some of the more tenuous arguments made by Farjoun and Machover, ones which would require empirical verification. Firstly, they conjecture that both $R$ and $Z$ are best fit to a gamma distribution. Like any argument for such a thing, this is done by arguing that the probability of both $R$ and $Z$ being less than $0$ is negligibly small. F and J make two adequate arguments for this. First, they argue that while there are in a capitalist economy always a high number of loss making firms, that these firms are overwhelmingly operating at very low levels of capital, and since our probability measure weighs firms according to the proportion of capital operating through them, this will massively skew the distribution to the positive half of the number line. The second argument they make is that even among the loss making firms, these firms are typically making a loss only after payment on the interest of capital that has been borrowed, and F and J explicitly decide to measure the capital of a firm \textit{prior} to these payments. Both of these arguments apply equally well to both $R$ and $Z$. \par 
Thus they assume that $R \sim Gamma(\alpha,\beta)$, and $Z \sim Gamma(\alpha',\beta)$. They assume without much argument that these two variables have the same scale parameter $\beta$, which I understand since in a certain sense $Z$ emulates $R$ as a constant proportion of $R$, but this would require empirical evidence and more of an actual logical argument to really justify. \par 
They then note that the value $e_0$ has, according to the empirical evidence they have seen, barely changes at all for long periods of time, and that the variable random variable $X$ has a nearly degenerate (i.e. nearly $0$ variance) distribution, also not changing with time, and also based on their empirical evidence. Furthermore they mention that this observed unchanging value of $e_0$ is $1$. It appears that more recent evidence has called this into question, however, and that it is actually closer to $\frac{1}{2}$. They then go on to argue that the initial two empirical observations necessitate the third, in a clean logical argument which I'll replicate here:
\begin{fact}
	Suppose $R \sim Gamma(\alpha,\beta)$, $Z \sim Gamma(\alpha',\beta)$, and $V\left( \frac{R}{Z} \right) = 0$. Then $R = Z$, and $\frac{E(R)}{E(Z)} = 1$.  
\end{fact} 
\begin{proof}
	If $V\left( \frac{R}{Z} \right) = 0$, then $\frac{R}{Z}$ is in fact a constant, call is $a$. Then $\frac{R}{Z} = a \implies R = aZ$. Thus it follows that $E(R) = aE(Z)$ and $V(R) = a^2V(Z)$. Writing these equations in terms of the parameters $\alpha, \alpha', \beta$, we have that
	\[ \frac{\alpha}{\beta} = a\frac{\alpha'}{\beta} \implies \frac{\alpha}{\alpha'} = a \]
	\[ \frac{\alpha}{\beta^2} = a^2 \frac{\alpha'}{\beta^2} \implies \frac{\alpha}{\alpha'} = a^2 \]
	From which it can be plainly seen that
	\[ a = a^2 \]
The shape parameter of a Gamma random variable must be greater than $0$, meaning that $a$ must also be greater than $0$. The only positive number $a$ such that $a = a^2$ is $1$. Thus $a = 1$, from which it follows that $\alpha = \alpha'$. 
\end{proof}
If it is truly the case that more recent empirical evidence shows this $a$ to not be $1$, this fact is not useless: it would in fact contrapositively imply that at least one of the other empirically justified claims must \textit{also} be false. Thus this fact implies that either the assumption that the scale parameters are the same is false, or the distribution of $X$ is not actually degenerate, or both, is what we are left to conclude. \par 
F and J were excited to have reached a definite estimate of $e_0$, since $e_0 \approx e^*$, meaning that it fixes the parameters of the normally distributed $\Psi$. That $\Psi$ is actually normal without much of any assumptions (aside from the decomposition of price into wages and profits) is fairly clear, and recall that with some extra minimal assumptions we arrived at the estimate that $E(\Psi) \approx 1+e^*$. With their estimate of $e_0 = 1$, they then conclude that $E(\Psi) = 2$. They then, with one more assumption about the model, proceed to derive an estimate for the standard deviation. \par 
The assumption is that the probability of $\Psi < 1$ is extremely small. This, to me, is one of the more innocent of the assumptions we've made so far. If $\Psi$ is consistently less than $1$, then that would mean that commodities are consistently being bought and sold at a loss in terms of the centralizing commodity. In the case of our particular model that central commodity is labour, so concretely it would mean that people are consistently buying commodities for lower prices than it would cost them to simply hire the manpower to produce the commodity themselves, provided they have the other raw materials and machinery. To highlight how absurd this would be, suppose this situation is true for oil, rather than labour. E.g. I buy plastic consistently for less than the cost of the oil required for refinement into plastic. If this were the case, then the firms producing oil would be consistently operating at a negative rate of profit, and very quickly nobody would be making oil anymore. If we assume this is true for labour, then that means that labour-power is consistently sold at a price lower than it's cost of production, which would mean that the workers selling their labour are consistently being paid with too little money to simply continue existing, and the working population would quickly die out from starvation, exposure, disease, etcetera. (If one insists on an argument which makes no mention of an ideal labour content of labour power, then we could instead simply say that the average workers themselves would quickly adjust their daily consumption to whatever lower quality of life that their wages actually allow. At the end of the day though, we would be back in the situation in which $\Psi$ is generally nonnegative.)  \par 
 This observation, independently of any arguments we might be interested in making about standard deviations, pokes at the very heart of the tautological nature of the model and the "anything" theory of value. \textbf{The labour theory of value, in statistical aggregate, is true, because the "anything" theory of value is true, because if it \textit{weren't} true, then the capitalist system would either have to quickly adjust itself in order to \textit{make} the theory true, or collapse into chaos.} \par 
 In any case, they then supply a rough estimate that the probability of $\Psi < 1$ is less than $\frac{1}{1000}$. This constrains the normal distribution to one such that the standard deviation is at most $\frac{e_0}{3}$, or in their particular case $\frac{1}{3}$. Thus they conclude that the specific price is normally distributed, with mean approximately $2$, and standard deviation approximately $\frac{1}{3}$. I find this reasoning entertaining but not nearly as important as they seem to think it is. The major results of their model, it seems to me, don't hinge on any particular shape to any of these distributions. 

