\section{A Stochastic Labor Theory of Value}
The main takeaway from Marx's metaphysical reflections and explorations of the three volumes of Capital are, in my opinion, that capital is in a \emph{real social fluid}. 
Just like a real \emph{material} fluid, it flows like in a river, or maybe even a circuit. The flows of capital intersect and criss cross. At some locations, the flow is wider and more intense, at others it's only a trickle. Just like a river, society subsists off of the flow of capital. In order to obtain the necessities of daily life, you have to either control part of the flow, or work against it to siphon off a small amount of it. Siphoning off fluid requires the application of labor - not exactly work in the physics sense, but rather socially necessary human labor. In any case, physical work is involved, and the aggregated physical labor of the working population acts as a massive motor force which is responsible for animating the flow in the first place. This bears repeating: In working against the flow to siphon off their subsistence, workers are \emph{producing} and perpetuating that flow in the first place. It is in a fetishistic reversal that we mistake the flow of capital as the flow of these sheets of paper called money, or more directly as something radically disconnected from social labor, something which is mystically bestowed it's power over us as individuals by natural law. This is much as the peasants who gave the king of a feudal regime his social power were eventually convinced that the king's power came from god. \par 
Just as a flow of water, the fluid of capital is understood to be in actuality an aggregate collection of moving particles. Just like an ideal gas cloud, the entity of the cloud, and it's observable dynamics, are the statistical results of the particles interactions. Crucial takeaways result from this conception:
\begin{itemize}
	\item[(1)] To effectively understand the basic mechanics of a capitalist system is to understand the basic statistical dynamics of the flow of capital.
	\item[(2)] The study of the flow of capital will heavily resemble the study of fluids and gas clouds, and so it will resemble the study of statistical mechanics.
\end{itemize}
\newunit{\auw}{auw}
\newunit{\aup}{aup}
\newunit{\dollars}{d}
I will assume for simplicity that our capitalist economy has a single currency, that is, a single medium of exchange which is traded for commodities. I will call this currency dollars, with units denoted by $d$. Our model economy will not be closed in the sense of having a fixed selection of commodities, as many economic models seem to be. However, the number of \textbf{firms} within our economy will be closed. Define the \textbf{firm space} for a model capitalist economy to be the set
	 \[ \mathcal{F} = \{ f_1,f_2,...,f_{n_F} \} \]
I wish to define in an objective way the amount of \textbf{capital} employed by a firm at a particular time $t$. I want to be very careful about this. Fix an interval of time, $T$. During this time, a firm will make use of a definite number of commodities as inputs to it's production (one of which is always human labour-power). This can be seen as a basket of goods, and this basket of goods itself, \textit{not it's dollar price or any other quantitative measure of these goods}, is the capital of a firm. That said, I want to view the capital of a firm as a number, and to do this, a numeraire must be specified. The simple thing to do would be to just use dollars. And this is what we will do, but this requires a disconnection in some sense from the objectivity of the situation, because while the inputs are a definite set of physical objects, their prices are the result of transactions occurring on the free market, on which anyone can sell anything at any price. In other words, prices are the product of a chaotic system of social decision making, and so to attach the numeraire of price to the commodities one must acknowledge the way in which our physical system of production is directly engaging with and reacting to a separate system of exchange, one in which endlessly complex social processes are taking place which can only be seen in purely physical terms as stochastic processes - processes \textit{which can only be observed statistically in aggregate}. Thus the price of a commodity must be seen as a random variable, however narrow or nearly degenerate that random variable might be. The upshot of this is that the \textit{physical collection of capital goods} is definite, but the \textit{value of that capital itself is the product of randomness}. 
\par 
However, the firm in question must purchase those materials that it employs in production as commodities, and despite their prices being random, that firm, in order to be using them, must at least have agreed to pay a definite price for those commodities. This agreed price of the commodities used during the interval $T$ (some of which will be partial uses, but this is fine, simply take the appropriate fraction of the price in those cases) is what we will define as the \textbf{capital} of the $i^{th}$ firm, which we will denote $K(f_i,t)$. (Where $t$ might as well be arbitrary or can be assumed known without specifying it, we will simply write  $K(f_i)$.) \par 
Note that since the number of firms is assumed finite, the \textbf{total social capital} $\mathcal{K}$ circulating in society is itself finite (at any specific moment in time). We define it as
\[ \mathbb{K} = \sum_{i =1}^{n_F}K(f_i) \]
We also use $\mathbb{K}$ do denote the total capital as a physical collection of goods, and distinguish this from the quantity $\mathcal{K}$. Using this capital function, a probability measure can be defined on firm space (one which changes with time). Simply define the probability of "drawing" a specific firm (or if one would rather, the \textbf{weight} of the $i^{th}$ firm) to be the proportion of capital in that firm versus the whole economy, i.e. 
\[ p(f_i) = \frac{K(f_i)}{\mathcal{K}}K(f_i) \]
The capital function itself then becomes a random variable on firm space, in the sense that
\[ P(k) = \sum_{i \in K^{-1}(k)}p(f_i) \]
Thus the probability of seeing a specific level of capital is precisely the proportion of capital which that bracket occupies out of the total social capital. The dynamics of a capitalist economy are precisely the dynamics of the distribution of capital. \par 
With these definitions, we have the following: The particles in our system are not firms, but capitals. What we have is essentially an ideal gas cloud, that is, a collection of identical particles in space. The space is a one dimensional line, made out of the firms (ordered arbitrarily), and the particles are the \textit{individual} capitals. Each capital is it's own entity. It's dynamics are the dynamics of capital, and the dynamics of the cloud are the dynamics of capitalism as a total system. This is not a perfect analogy. Capitals are created moment to moment by the prices agreed upon, and so the total social capital is slowly changing over time - particles are winking in and out of existence. Additionally, the total social capital need not be a whole number - single "unit" capitals might split in half and move in different directions. Nonetheless, the analogy will be extremely useful to keep in mind. \par 
Continuing on the gas cloud analogy, while the capitals might be the same, they by no means have a uniform distribution in space, that is, among the firms. At any moment in time, $70$ percent of all $\mathbb{K}_G$ many particles might be occupying the same location (i.e. one single firm might be dominating production to the point where they are employing seventy percent of all of the capital). In the situation that individual particles have the same mass, the location where population is most dense is the location that will have the most effect on gravity. The same would be true of the distribution of charge in an electrodynamical system, and the same is true of capital in a capitalist system. The dynamics of the economy revolve and are dictated by the firms operating with the most operating capital, and defining a probability measure via the distribution of capital means that all future random variables defined will be contingent on this distribution of capital, being influenced by firms proportionally based on the proportion of capital within the firm itself. It should be emphasized that there is no causal claim being made here. This is a choice, not a guess, of what quantities are believed to be central in the sense that we are concerned how the capitalist economy relates to these central quantities. In the case of capitalism and the dynamics of capital, this choice is obvious. As the reader will see shortly, it won't always be that way.
 \par 
Central to our human economy is the labour which participates in it. During the period $T$ specified above, there is a portion of the physical goods which are purchased by the workers in exchange for bills they receive from their employer. Note here I am stressing the \textit{physical goods} as the wages, not the actual money, just as I did with capital. For now it suffices and is crucial to view the wages as a definite collection of physical goods and services, unquantified. (The remainder of the goods produced are the \textbf{surplus}, and the quantification of this surplus will be profit.) \par 
From the perspective of the capitalist, human labour power is purchased like any other commodity. However, it is a very peculiar commodity in the following sense. Suppose I'm a capitalist, and I want to enter the cardboard box making industry. How would I go about this? I would ask around how \textit{cardboard boxes are already made}. That is to say, I would look at how my competitors are doing it, and either copy them, or copy them and add on some very small innovation to what they were doing. Either way, we can conclude that commodities generally have a "standard method of production". There is at any moment in time $t$, a standard set of procedures, machines, quantities of inputs, style of work being employed, etcetera, which is remarkably stable and changing usually very slowly. Perhaps more realistically there are two or three methods, depending on whether we're talking about a global system or a national system, but that is manageable by taking averages (we will return to this in a bit). \par 
This is the case for all commodities \textit{except}, arguably, for labour power, since people reproduce themselves \textit{outside} of the capitalist economy. The reproduction of labour power \textit{does} require labour, but from the perspective of capital this labour is not productive of profit, because if it were, then that would mean there is a capitalist firm selling the product of this reproductive labour for profit, and slavery is illegal. Thus it must be acknowledged right away that the system of humans is actually two interconnected systems of labour organization. In the \textbf{domestic economy}, humans take care of themselves and perhaps a few of their family members, reproducing their labour power. From the perspective of a profit seeking \textbf{capitalist economy}, the domestic economy is simply the non-profit industry in which humans reproduce units of labour power for sale on the market as inputs to production.  \par 
I say arguably because Marx himself would disagree. In Capital, Marx defines precisely this quantity, which he calls the \textbf{means of subsistence} - the ideal price of reproduction of labour power, determined by society's average daily (or weekly or whatever the period in question is) consumed goods. Just as a realistic model must swear off from ideal prices, we should do the same here and reject an ideal wage. Nonetheless we will denote this proposed basket of goods $\mathbb{V}_0$, for later comparison with Marx's own model. 
Despite not having a standard method of production, the production of labour power, like all other commodities, \textit{does} consume, for each period of time $T$, a fixed collection of commodities, which we will denote $\mathbb{V}$, and that collection of goods is precisely what I defined earlier as wages. These wages are paid by the firm, and must be equal to a portion of the capital. In the spirit of Marx, we will call this the \textbf{variable capital} of the $i^{th}$ firm. The remaining basket of non-labour goods used in production during the period $T$ is denoted $\mathbb{I}$, and is called the \textbf{constant capital}. Thus, in a loose sense (these are physical collections of goods, not numbers to be added) we can say 
\[ \mathbb{K} = \mathbb{V}+\mathbb{I} \]
There is one more physical collection of commodities which we should define and give a special symbol for - the commodities which are actually produced by the capitalist economy after the period $T$ has ended. This we denote $\mathbb{P}$. \par 
Before turning to the rate of profit, we should introduce two more spaces. For the same fixed period $T$, there are a definite number of transactions which occur, in which money is exchanged for commodity. We make no reference or restriction to commodity type. If I pay seventy dollars my groceries at he store, then that is a single transaction, and the bundle of groceries I leave the store with are a single commodity. We denote \textbf{market space} as an indexing of these transactions:
\[ \mathcal{M} = \{c_1,c_2,...,c_{n_M}\} \]
The only transactions left out of this set are those in which the commodity sold is labour power. For this, we fix a unit of measurement of labour, usually as the period length $T$, but for the sake of conceptual simplicity one can think in terms of hours. The reason $T$ makes for a good unit is that it means that the total amount of labour sold is always precisely the number of workers in the capitalist economy (assuming each worker sells the same amount of labour, which is quite a strong claim to make, but even if it is off, it will still be the case that it is a close approximation.) Then for the period $T$, the number of units of labour power sold is definite, and we index these units in \textbf{labour power space}:
\[ \mathcal{L} = \{l_1,l_2,...,l_{n_L}\} \]
We will also denote $N$ as the total number of workers in he economy, and subject to choice of units of time, in the case of unit labour time $T$, we will often substitute $N$ for $n_L$. The probability measure on labour power space will simply be the uniform measure:
\[ p(l_i) = \frac{1}{n_L} \]
Let $S$ denote the total number of dollars paid for all of this labour power. From $S$ as well as the total labour done in the period $T$, we will define a unit price from which we will measure prices in general. Define the \textbf{average unit wage} to be the quantity:
\[ 1 \auw = \frac{S}{n_L} \dollars \]
We will in general measure prices in units of $\auw$, rather than simply dollars. If, on average, it costs $16$ dollars to purchase one worker hour, and I pay $2$ dollars for coffee, then the price of the coffee is $.125\auw$. That is to say, what I paid for a cup of coffee is an eighth of what it would cost, on average, to purchase an hour of labour (or a unit of labour, more generally). \par 
With $\auw$ defined, we can begin defining the main random variables in question. First of all, for each unit of labour power sold in the economy, $l_i$, a particular wage is paid, in dollars. This wage, expressed in $\auw$, is the \textbf{wage} paid for the $l_i^{th}$ unit of labour, and is denoted $W(l_i)$. Thus the wage is a random variable on labour-power space. Note that because we are measuring in $\auw$, that it is always the case that $E(W) = 1$.
\begin{proof}
	Enumerate each particular wage amount $w_1,w_2,...,w_k$. Then
	\begin{align}
		E(W) = \sum_{i=1}^k w_i p(W = w_i) &= \sum_{i=1}^k w_i \sum_{j \in W^{-1}(w_i)}p(l_j) \\
			&= \sum_{i=1}^k w_i \sum_{j \in W^{-1}(w_i)}\frac{1}{n_L} \\
			&= \frac{1}{n_L}\sum_{i=1}^k w_i |W^{-1}(w_i)| \\
			&= \frac{1}{n_L}\sum_{i=1}^{n_L}w_i \\
			&= \frac{S}{n_L} = 1 \auw
	\end{align}  
\end{proof}
Now, let us turn back to market space. We still need to define the probability measure for this space. Just as with firm space, we want to view market space as a distribution not of the transactions themselves, but of more indivisible units which just happen to occupy certain transactions, as a particle would occupy a location. Here, though, the choice of how to do this is not obvious, but somewhat arbitrary. I'm going to give the definitions first, and then talk about them once the important random variables to the discussion have been laid out. \par 
The individual equally weighted particles in market space will be individual units of labour. Thus, transactions will be weighted proportionally to how many units of labour went into the physical objects being traded. To be clear, I've already mentioned that it is reasonable to assume a standard method of producing any commodity other than labour power in a capitalist economy (the claim would \textit{not} be true necessarily for modes of production other than capitalism). Given that there is a standard method, there is also a standard basket of commodities which is employed in the creation of whatever is defined as a "unit" of the commodity in question (the choice of which is arbitrary). 
\\
\\
elaborate on this
\\
\\
Let $\lambda(c_i)$ denote the \textbf{labour content} of the commodity $c_i$. Note that since the set of transactions is finite, the total collection of commodities is itself a definite basket, which I'll denote $\mathbb{C}$. Note that $\mathbb{C}$ is a very similar basket to the basket of all commodities produced $\mathbb{P}$, but not necessarily identical. Some commodities produced will not be sold until later, and some commodities sold were produced during an earlier period. Nonetheless, they will usually be very similar to each other, so for the record I'll write:
\[ \mathbb{C} \approx \mathbb{P} \]
Baskets of commodities have a total labour content as well. I'll employ some polymorphism and write $\lambda(\mathbb{C})$ to denote the total labour content of all goods and services sold during the period $T$ (although since the commodities involved in the transactions might be collections of things bought at the same time, the distinction is mostly meaningless). From this, we can define a probability measure on market space. The probability of "drawing" a particular transaction will be simply the proportion of labour which that commodity being traded for represents relative to the total labour being traded:
\[ P(c_i) = \frac{\lambda(c_i)}{\lambda(\mathbb{C})} \]
At this point, the labour content of a transaction becomes a random variable, $\Lambda$, but a fundamental one in the sense that I am now choosing a specific perspective on the capitalist economy. By defining the probability measure on market space in terms of the labour content of commodities traded, I am now choosing to specifically concern myself with how the capitalist system interacts with the distribution of labour in producing the commodities being traded. The model is similar to the model for firm space: particles in firm space are individual, equally weighted "capitals", while particles in market space are individual, equally weighted "units of labour-power". We are now discussing a system in which the central concern is the dynamics of markets and capitalism \textit{relative} to the distribution of labour. The degree to which this choice is arbitrary will be discussed shortly. \par 
Before that discussion however, we should lay out the whole model relative to this labour basis. The random variable $\Pi$ will denote the actual price of a commodity in drawn in market space weighted according to labour proportion, measured in units of $\auw$. If $c_i$  indexes my purchase of a coffee from earlier, then $\Pi$ is not equal to $2$ dollars, but rather $.125 \auw$. I'll also write $\Pi = \pi(c_i) = 0.125$, i.e. use lower case $\pi$'s to denote particular values that the random variable $\Pi$ takes on. \par 
Finally, we denote the central random variable of interest pertaining to market space and labour-power space, the \textbf{specific price} $\Psi$ of a commodity. This is defined
\[ \Psi = \frac{\Pi}{\Lambda} \]
Thus the specific price of a commodity is the arrived at price of the commodity as a number of units of labour power which could be purchased at that same price (on average) versus the actual number of units of labour power which are actually needed in order to create the commodity being traded. \par 
This is an extremely provocative quantity. Suppose I pay for a commodity $c_i$, and that $\Psi < 1$. What does this mean? It means that, relative to the labour which went into creating this commodity, I struck a bargain; the amount of money I spent on $c_i$ would generally not be sufficient to the amount of labour I would need to purchase in order to produce the commodity myself. Conversely, if $\Psi > 1$, then from the standpoint of labour, I've been ripped off; I paid enough that the person who sold $c_i$, provided that they have the other necessary materials, could buy enough labour power to reproduce $c_i$ \textit{plus} some extra surplus. \par 
This means that our model has been specifically built in order to evaluate capitalist profit from the standpoint of a particular type of profit - profit of labour time. $\Psi > 1$ might mean that the merchant has profited in terms of labour, but they might have been ripped off in terms of, say, oil. To fully understand this framing, let's go through what would change to reframe our model in terms of oil rather than labour. \par 
First, instead of labour power space, we would instead define \textit{oil space}, 
\[ \mathcal{O} = \{o_1,o_2,...,o_{n_O} \]
in which we've specified a unit of oil, say perhaps gallons, and declared $\mathcal{O}$ to be an indexing of all gallons of oil sold during a fixed interval of time $T$. We can then declare a unit of measurement for price in terms of gallons of oil. Suppose the total price paid for oil is $S$, then we can define the average unit price by
\[ 1 \aup = \frac{S}{n_O} \]
We can then declare the oil price random variable $G$ to be the price of a particular oil unit measured in $\aup$, from which it can be shown that $E(G) = 1$, just as before for wages. We can then relate this denomination back to market space and define prices to be measured in units of $\aup$, that is, measuring in dollars, but not so much \textit{in} dollars, so much as in the number of gallons of oil that a certain dollar amount could purchase, on average. We can also measure the oil content of any commodities in the same sense that we measured the labour content, and denote it, say $O$, and then define specific price as the variable which is price (in gallons of oil purchasable by a given price) relative to the average cost of a gallon of oil. From this, we have built up, seemingly, the exact same model, except now we are concerned with the dynamics of capitalism relative to oil, as opposed to labour. There is a problem, however. \par 
The problem is what I've already mentioned: oil is produced as a commodity, whereas labour power is not. When we were discussing market space in relation to labour power, oil was presumably a commodity like any other, whose purchases and sales were indexed in market space. Thus oil was not \textit{absent} from the system, it just wasn't singled out as peculiar. The question we must ask is this: can the same be said of labour power in our new oilcentric system? Remember that the central assumption to this model is the definite existence of a meaningful oil-content of any commodity bought or sold. Well, what is the oil content of a unit of labour-power? Such a number must be assumed in order to render labour power sales in market space, but as we noted before, labour power is \textit{not} produced systematically in the same sense that other commodities are. People live wildly different lives from each other, there is no "mostly standardized method of existence" in the same sense that there is a "mostly standardized method of producing pencil lead". Such a number can be assumed (and Marx \textit{did} assume just such a thing), and in some sense it \textit{does} exist, in an average sense. But it requires it's own extremely dubious simplification of reality, a complete equivocation between the domestic economy of people, and the capitalist economy of commodity production. \textit{The only way to incorporate labour power into an oilcentric model of capitalism is to cover up and ignore the peculiarity of labour power.} \par 
This is not to say that an oil-centric model of capitalism can't be useful in it's own right. It has things to say, valuable things even. But the only way to evaluate capitalism in it's interaction with the humans labouring and benefiting from the system is to distill labour out of it. In order to understand capitalism \textit{relative to the human society from which it spawned}, is to single labour-power out as a special and fundamental substance which the dynamics of capitalism play out \textit{in relation to}. \par 
Thus, I will proceed to make claims about relations between prices and labour contents, and claims about capitalism's dynamics relative to labour, and those claims should be subject to the scrutiny of empirical scientific testing and verification. But the "labour theory of value" will not be \textit{proven} scientifically, nor should it be. It is not a scientific question, but a philosophical one. This model itself has a hidden universal quantifier behind every claim it makes, and this is something that I want to pay special and careful attention to as I continue laying the model out. \par 
With that said, let's return to market space. I defined the capital of a firm earlier, but at that point I did so in terms of dollar value. Now, I am considering a system "relativized" to labour, and as such I want to measure capital in units of $\auw$. I want to next turn to the rate of profit. Within the period of time $T$, a firm $f_i$ will produce some physical output of commodities $\mathbb{P}_i$, employing a total physical capital $\mathbb{K}_i$, which can be divided into wages and materials $\mathbb{V}_i$ and $\mathbb{I}_i$ respectively. The firm sells it's produce at a price $\pi(\mathbb{P}_i)$, constituting the revenue, and, as anyone who takes a business class will tell you, it's profit is the difference between revenue and cost, which is of course $\pi(\mathbb{I}_i) + \pi(\mathbb{V}_i)$. Now, for the purposes of rigour, it must be emphasized that these values of $\pi$ are not necessarily the outputs of the same function $\pi$ defined on market space. Remember that the random variable $\Pi$, and it's specific cases $\pi$, pertain specifically to commodities bought and sold during the fixed interval $T$, while the some of the commodities produced will likely be sold at a future interval, and some of the materials and labour power purchased were purchased prior to the interval beginning. Nonetheless, from the perspective of the firm, the profit rate is obviously the ratio of profits relative to the capital invested:
\[ r_i = \frac{\pi(\mathbb{P}_i) - \pi(\mathbb{I}_i) - \pi(\mathbb{V}_i)}{\pi(\mathbb{K}_i)} \]   
It should immediately be noted that $\pi(\mathbb{K}_i)$ is merely a fancy new way to express what I defined in the beginning, the capital of the $i^{th}$ firm $K(f_i)$. Or is it? The ambiguity here points to an ambiguity which was never resolved when I defined this random variable initially, which is that while the capital goods used in a period of production are definite, the numerical price of these goods is slippery. The simplest thing to do, and what we implicitly did back then, was to simply define $K(f_i)$ to be the number of dollars (expressed in $\auw$) spent by the firm whenever they bought the constituent goods. This is the only reasonable thing to do, and what I will do for the prices of the other bundles in this equation $\pi(\mathbb{P}_i), \pi(\mathbb{I}_i)$, and $\pi(\mathbb{V}_i)$. In any case, we've now defined the rate of profit $R$ of the $i^{th}$ firm, as a random variable on firm space. We've also defined the constant and variable capital $I$ and $V$ as random variables, via $I(f_i) = \pi(\mathbb{I}_i)$ and $V(f_i) = \pi(\mathbb{V}_i)$, as well as the price of produced goods $Pr(f_i) = \pi(\mathbb{P}_i)$. We also have the identity that 
\[ K = I+V \]
And we can now reexpress the rate of profit as a random variable defined in terms of these:
\[ R = \frac{Pr-I-V}{K} \]
Before moving on, note that while the numerical value of $K(f_i)$ will necessarily change when we go from measuring in dollars to measuring in $\auw$, the profit rate will remain the same regardless of the commodity we are relativizing our model to. The reason for this is that while the capital of a firm has a physical unit associated with it (namely either dollars $\dollars$ or average unit wages $\auw$), these units cancel out in the fraction defining the rate of profit, rendering the rate of profit unitless. Moreover the conversion from $\dollars$ to $\auw$ would equate to multiplying each term by the unitless factor $\frac{n_L}{S}$, all of which cancel out immediately just as the units do. Thus \textit{profit rates remain the same regardless of the commodity chosen to act as numeraire.} 
Suppose that for a period $T$, beginning at time $t_0$, a firm has the profit rate $R(f_i) = r_i$, and similarly a constant, variable, and working capital $j_i,v_i$, and $k_i$ respectively. Suppose the period $T$ is relatively small. In this case, the technical details of production, habits of consumption, and state of the class struggle (i.e. levels of wages) are not likely to change much. As a result, the prices in the equation for rate of profit are not likely to change much, and the rate of profit can reasonably be expected to stay approximately the same in the \textit{next} period, that is, from time $t$ to time $t+T$. Assuming that the profit of the previous period is reinvested as capital in the next period, we have that \begin{align}
	K(f_i,t_0+T) &= K(f_i,t_0)+Pr(f_i,t_0)-I(f_i,t_0)-V(f_i,t_0) \\
			&= K(f_i,t_0) + K(f_i,t_0)R(f_i,t_0) \\
			&= K(f_i,t_0)(1+R(f_i,t_0))
\end{align}
It's plain to see from this that assuming a decision to reinvest profits results in the rate of profit becoming an exponential growth factor for capital, but let's dig a bit deeper. The total change in capital then becomes:
\begin{align}
	\Delta K = K(f_i,t_0+T)-K(f_i,t_0) &= K(f_i,t_0)(1+R(f_i,t_0))-K(f_i,t_0) \\
	&= K(f_i,t_0)((1+R(f_i,t_0))-1) \\
	&= K(f_i,t_0)R(f_i,t_0)
\end{align}
If we fix $T=1$, i.e. we view the period $T$ as our fundamental unit of time (perhaps weeks, or years), then this change in $K$ can be seen as average slope of the capital as a function of time. For small $h$ then, we can approximate the capital as a function of time as a straight line, i.e. we assume that for general $t$ in a small neighborhood of $t_0$, 
\[ K(f_i,t) \approx m(t-t_0)+b \]
where $m = \Delta K = \frac{K(f_i,t_0+T)-K(f_i,t_0)}{T}$ and $b = K(f_i,t_0)$, the capital at time $t_0$. By substituting $K(f_i,t_0)R(f_i,t_0)$ for $\Delta K$, we have that for $t$ nearby $t_0$.
\begin{align}
 	K(f_i,t) &\approx \Delta K(t-t_0)+K(f_i,t_0) \\
 			&= K(f_i,t_0)R(f_i,t_0)(t-t_0) + K(f_i,t_0)  
\end{align}
Thus for a small $h>0$, $t_0+h$ will be very near $t_0$, and we can write
\begin{align}
	K(f_i,t+h) &\approx K(f_i,t_0)R(f_i,t_0)(t_0+h-t_0) + K(f_i,t_0) \\
	&\implies K(f_i,t_0+h)-K(f_i,t_0) \approx K(f_i,t_0)R(f_i,t_0)h \\
	&\implies \frac{K(f_i,t_0+h)-K(f_i,t_0)}{h} \approx K(f_i,t_0)R(f_i,t_0)
\end{align}
But the left-hand side clearly approaches the derivative of $K(f_i)$ with respect to $t$ as $h$ approaches $0$, while the right-hand side is constant. We've thus isolated the relationship between the capital of a firm $f_i$ and the rate of profit of that firm. If we assume that the capitalist firms are constantly reinvesting their profits into production after each period $T$, then we obtain a discrete process which, when "smoothed out", yields curves for $K(f_i,t)$ and $R(f_i,t)$ as functions of time $t$ which are deterministically related by the differential equation
\[ \frac{dK(f_i,t)}{dt} = R(f_i,t)K(f_i,t) \]
If we assume the rate of profit to be a constant with time, simply $R(f_i)$, then the solution to this as an initial value problem is 
\[ K(f_i,t) = K(f_i,t_0)e^{R(f_i)t} \] 
The "rate of profit" then, as capitalists and economists tend to think about it, is not as simple as simply the rate of change of the capital over time. It is in fact a constant of proportionality to an assumed exponential growth in capital over time. Such an exponential relationship would remain true in principle if the capitalist only reinvested a certain portion of their profits after each period. \par 
Note that this exponential relationship between the rate of profit and capital is not definite, in two senses. In one sense, it is contingent on the assumption that a capitalist will reinvest a portion of profits to expanding production. In another sense, our model is not assumed to be a smooth or continuous process, but is instead a stochastic process which updates after specified intervals of time. In other words, this discussion is only for the sake of intuition on how the rate of profit and capital are going to interact within a normally functioning capitalist system. \par 
Before moving on to some of the other random variables we wish to define on firm space, let's consider a basic fact about the expected value of the rate of profit, one which will in some sense vindicate our choice of defining proportions of capital as the probability measure. Typically, the expected value of a random variable $X$ on a probability space is defined by summing or integrating along the range of the random variable $X$. That is, if $Range(X) = \{x_1,x_2,...,x_m\}$, and the probability space is $\Omega = \{1,2,...,n\}$, then
\begin{align}
	E(X) = \sum_{i=1}^m x_i P(X = x_i) &= \sum_{i=1}^m x_i \sum_{j: X(j)=x_i} P(j) \\
	&= \sum_{i=1}^m \sum_{j: X(j)=x_i} x_iP(j) \\
	&=  \sum_{i=1}^m \sum_{j: X(j)=x_i} X(j)P(j) \\
	&= \sum_{i=1}^n X(i)P(i)
\end{align} 
Thus, in the case of a finite sample space, summing over the range of $X$ the numbers obtained from taking each element and multiplying by the probabilities of realizing that element is the same as simply summing directly over the sample space the numbers obtained from the random variable output times the probability of that particular outcome. Using this identity for expectation, we can obtain a striking expression of the expected rate of profit:
\begin{align}
	E(R) &= \sum_{i=1}^{n_f} R(i)P(i) \\
		&= \sum_{i=1}^{n_f} \frac{Pr(i)-I(i)-V(i)}{K(i)}\frac{K(i)}{\sum_{j=1}^{n_f}K(j)} \\
		&= \frac{\sum_{i=1}^{n_f} Pr(i)-I(i)-V(i)}{\sum_{j=1}^{n_f} K(j)} \\
		&= \frac{\sum_{i=1}^{n_f} Pr(i) - \sum_{i=1}^{n_f} I(i) - \sum_{i=1}^{n_f} V(i)}{\sum_{i=1}^{n_f} K(i)} \\
		&= \frac{\pi(\mathbb{P}) - \pi(\mathbb{I}) - \pi(\mathbb{V})}{\pi(\mathbb{K})}
\end{align}
Thus the expected rate of profit is simply the global rate of profit: the total profit across the entire economy divided by the total capital, all expressed in terms of the observed actual prices. This number we will refer to as the \textbf{general rate of profit}, and denote $r_g$ 
\[ r_g = \frac{\pi(\mathbb{P}) - \pi(\mathbb{I}) - \pi(\mathbb{V})}{\pi(\mathbb{K})} = E(R) \]
Next we would like to inspect more closely the division of the total capital $K$ into it's two components $I$ and $V$. First note that our division is one relative to labour. Just like I noted earlier, this choice of labour costs specifically to single out here is technically arbitrary. The choice could have just as easily been to define $V$ to be capital costs spent on oil, rather than labour power. I'll return to this point later. For now, let's inspect how $V$ changes in relation to $K$ and $R$. Define the random variable $Z$ by
\[ Z = \frac{V}{K} \]
That is to say, $Z$ is the proportion of wages relative to the total capital. Towards an understanding of $R$, I supposed earlier that all profits were reinvested as capital. When I did that, I paid no attention to the \textit{composition} of those profits. Perhaps the capitalist uses their profits to invest in new machinery, which requires less labour. In that case, even if all profits return as capital, the proportion $Z$ could very well be smaller than it was before. Let's suppose that the capitalist simply reinvests in the same composition as before, that is to say, that
\[ Z(f_i,t_0+T) = Z(f_i,t_0) \]
i.e.
\[ \frac{V(f_i,t_0+T)}{K(f_i,t_0+T)} = \frac{V(f_i,t_0)}{K(f_i,t_0)} \]
But immediately from this we obtain
\[ V(f_i,t_0+T) = Z(f_i,t_0)K(f_i,t_0+T) \]
Thus, subtracting the original wages $V(f_i,t_0)$ and applying the fact that $V = ZK$ by definition, we get 
\begin{align}
	V(f_i,t_0+T) - V(f_i,t_0) = \Delta V &= Z(f_i,t_0)K(f_i,t_0+T) - V(f_i,t_0) \\
	&= Z(f_i,t_0)K(f_i,t_0+T) - Z(f_i,t_0)K(f_i,t_0) \\
	&= Z(f_i,t_0)\Delta K \\
	&= Z(f_i,t_0)K(f_i,t_0)R(f_i,t_0)
\end{align}
From here the argument proceeds identically to $R$. Considering $T$ as our unit time, and assuming $0<h<T$ and that $T$ is sufficiently small that the curves $V(f_i,t)$, $K(f_i,t)$ and so forth can be approximated as straight lines (and rendered as continuous growth), we have that $\Delta V$ functions as the slope of this line, and obtain 
\[ V(f_i,t) \approx Z(f_i,t_0)K(f_i,t_0)R(f_i,t_0)(t-t_0)+V(f_i,t_0) \]
Plugging in $t_0+h$ for $t$ and subtracting $V(f_i,t_0)$ from both sides and dividing by the resulting $h$ thus gives
\[ \frac{V(f_i,t_0+h) - V(f_i,t_0)}{h} \approx Z(f_i,t_0)K(f_i,t_0)R(f_i,t_0) \]
Taking the limit as $h \to 0$ thus yields the approximate law
\[ \frac{dV(f_i,t)}{dt} = Z(f_i,t)R(f_i,t)K(f_i,t) \]
The truly striking equation however is given when we realize that $R(f_i,t)K(f_i,t)$ is the derivative of $K(f_i,t)$ by a subset of the same assumptions we made to get here:
\[ \frac{dV(f_i,t)}{dt} = Z(f_i,t)\frac{dK(f_i,t)}{dt} \] 
Thus we see that the "natural" behavior of $V$ and $K$, under the assumption that the proportion of wages to constant capital remains unchanging, is for the variable capital to "mimic" the total capital, up to $Z$ a constant of proportionality. \par
Next let's consider the expected value of $Z$, similarly to how we did for $R$:
\begin{align}
	E(Z) &= \sum_{i=1}^{n_f} Z(f_i)P(f_i) \\
		&= \sum_{i=1}^{n_f} \frac{V(f_i)}{K(f_i)} \frac{K(f_i)}{\sum_{j=1}^{n_f} K(f_j)} \\
		&= \frac{\sum_{i=1}^{n_f} V(f_i)}{\sum_{j=1}^{n_f} K(f_j)} \\
		&= \frac{\pi(\mathbb{V})}{\pi(\mathbb{K})} 
\end{align}
Thus the expected value of $Z$ is the global ratio of total wages to total capital in circulation. \par 
I want to stop here and digress a bit to connect these variables to what Marx was doing in capital. Marx's central variables were $S$, the surplus value, $V$, the variable capital, and $C$, the constant capital. In order to make sure all of our objects of interest have unique symbols representing them, I'm going to denote these $S_M$, $C_M$, and $V_M$. These numbers were all values, that is to say, they were conceptually speaking amounts of time. Recall that Marx operates in an idealized world in which there is a definite "average labour content of labour power", which he called the means of subsistence. This number could be determined globally or locally, and relative to any interval of time. Globally, this would be, for a period $T$, approximately equal to the total labour content of all of the commodities produced, i.e. $V_M^g = \lambda(\mathbb{V})$. Such a number certainly has meaning, even in our own model. This number for us would represent something close to the total work that the workers must do each day in order to reproduce their way of life for another period $T$. One can only say something close, not exactly the same, since of course a good portion of the work that must be done was already done at an earlier period, or is work which, while necessary, can be put off to a later date without having any effect on people's material lives. Nonetheless, as a number which fluctuates from period to period, one can easily imagine an average or "ideal" amount. This number obviously \textit{exists} in some sense. What is the problem, then? The problem is that this number, divided by the number of laborers, while representing an average amount time one has to work in order to "pull their weight", but this number likely differs in the extreme from what any particular worker actually consumes for themselves. This is especially true if we are speaking about a global capitalist economy, since the consumer consumption habits of a North American are wildly different from those of a sweatshop worker in Bangladesh. Here, such an average certainly exists, but exists in a "middle" which likely corresponds to an extremely small population even remotely closely. \par 
At the level of a firm, the number $\lambda(\mathbb{V}_i)$ would represent something similar. It would be the number of units of labour power the workers \textit{specifically at that firm} must perform to replicate \textit{their} way of life. Our $V$, in contrast, while similar as a quantity to Marx's refers to the actual prices paid for these commodities - $V_M = \pi(\mathbb{V}_i)$. Similar for $C$: to connect Marx's model to ours would be to say $C_M = \lambda(\mathbb{I}_i)$, while in contrast we have the variable $I = \pi(\mathbb{I}_i)$. Finally, let's consider Marx's surplus value $S_M$. Globally, it is the total labour value pocketed by the capitalist class which is in excess of the labour required for the workers to reproduce their way of life. It's tempting to just conclude that this labour content is embodied in the value $\lambda(\mathbb{P})$, that is, the total labour content of the bundle of all commodities produced during the period $T$. But this ignores the fact that much of this bundle is the product of both living \textit{and} dead labour. We cannot forget that this collection $\mathbb{P}$ is the product of living labour being added onto the raw materials $\mathbb{I}$, which were by definition produced prior to this period. (If a firm produces a unit of these raw materials, then that unit would no longer be part of the raw materials bundle. Instead the product made from it would appear in the bundle $\mathbb{P}$. To put the raw material in $\mathbb{P}$ \textit{and} $\mathbb{I}$ would make the identity claimed following this parenthesis invalid. So this is really a clarification that should have been provided earlier.) Thus the total labour time produced during the period $T$ is $\lambda(\mathbb{P})-\lambda(\mathbb{I})$. From this, the connection to our model would be that 
\[ S_M^g = \lambda(\mathbb{P})- \lambda(\mathbb{I}) - \lambda(\mathbb{V}) \]
The same argument would be true at the level of the firm: 
\[S_M(f_i) = \lambda(\mathbb{P}_i)- \lambda(\mathbb{I}_i) - \lambda(\mathbb{V}_i) \]
From these variables, Marx defined three peculiar ratios. These he didn't give consistent letters to, so I'll take my own liberties. First there was the rate of profit:
\[ R_M \equiv \frac{S}{C+V} = \frac{\lambda(\mathbb{P})-\lambda(\mathbb{V})}{\lambda(\mathbb{K})} \]
Or, at the level of the individual firm,
\[ R_M(f_i) \equiv \frac{S}{C+V} = \frac{\lambda(\mathbb{P}_i)-\lambda(\mathbb{I}_i) - \lambda(\mathbb{V}_i)}{\lambda(\mathbb{K})} \]
This looks remarkably similar to our definition of the rate of profit. Marx would claim that they are not just similar, but identical. The reason for this is that Marx made the following claim, the central column of his labour theory of value:
\begin{claim}[Marx's Labour Theory of Value]
	For a capitalist economy in which supply and demand are in constant equillibrium, there exists a constant $\psi_0$ such that for all commodities $C$, 
	\[ \frac{\bar{\pi}(C)}{\lambda(C)} = \psi_0 \]
where $\bar{\pi(C)}$ denotes the \textbf{ideal price} of $C$, that is, the nonrandom price which $C$ would take on, assuming that supply and demand were constantly equal throughout the capitalist economy.
\end{claim}
In other words, Marx claimed that priced would be identical with labour time up to some constant of proportionality converting units of dollars to units of time units, if supply and demand were in equilibrium. The coherence of this claim hinges on the existence of ideal prices, in the sense described above. We make no such claim, but for the sake of comparison let us assume for a moment that these prices did exist, and that our economy observed whatever conditions necessary that prices were ideal and deterministic, i.e. let us assume that $\bar{\pi}(C) = \pi(C)$. Then under the assumption of Marx's claim, his rate of profit is indeed identical to ours, since we can use this identity to substitute $\pi$ for $\lambda$: 
\[ R_M(f_i) \equiv \frac{\lambda(\mathbb{P}_i)-\lambda(\mathbb{I}_i) - \lambda(\mathbb{V}_i)}{\lambda(\mathbb{K})} = \frac{\pi(\mathbb{P}_i)\psi_0-\lambda(\mathbb{I}_i)\psi_0 - \lambda(\mathbb{V}_i)\psi_0}{\lambda(\mathbb{K})\psi_0} = \frac{\pi(\mathbb{P}_i)-\pi(\mathbb{I}_i) - \pi(\mathbb{V}_i)}{\pi(\mathbb{K})} = R(f_i) \]
Now, we don't have such a claim, but we do have a random variable $\Psi = \frac{\Pi}{\Lambda}$. In a little bit, we will obtain a result about $E(\Psi)$ which will allow us to show that, under a few assumptions, 
\[ \frac{\lambda(\mathbb{P}_i)-\lambda(\mathbb{I}_i) - \lambda(\mathbb{V}_i)}{\lambda(\mathbb{K})} \approx E(R) \]
[Need to show]
Marx also defined the \textbf{organic composition of capital} to be the ratio of constant capital to variable capital measured in labour time. Globally, this would mean
\[ O = \frac{C}{V} \equiv \frac{\lambda(\mathbb{I})}{\lambda(\mathbb{V})} \]
To recall Marx in some precision here, recall that Marx really defined three versions of "the composition of capital" back to back to back, late in Volume I. The \textbf{technical composition of capital} is what Marx defined as the actual material goods ratio of materials per worker. This is not a number that can be expressed purely numerically. If one widget machine and four magnets is enough to supply three workers for an hour of work, then the technical composition here is "one widget machine and four magnets per three worker hours". In other words, the technical composition of capital is the actual organizational distribution of capital goods per worker per unit of labour time. Next, Marx defined the \textbf{value composition of capital} as the numerical labour content ratio of the technical composition. That is to say, take the bundle of commodities to accommodate some number of units of labour-power, take the labour contents of both, and express that as a numerical ratio. (Remember, Marx is assuming that labour-power has a definite labour-content just like every other commodity.) Finally Marx defined the organic composition of capital as "the value composition of capital, in so far as it is determined by its technical composition and mirrors the changes of the latter." From this, I am interpreting that he is simply saying to take the value composition obtained from the technical composition, and render it as a function of a now changing with technology and innovation over time. Since our entire model is assumed dynamic, I feel comfortable referring to our number above as the organic composition, and not the technical or the value composition. It is the whole package. We don't have a variable yet which is directly analogous to $O$, but we do have the random variable 
\[ Z = \frac{V}{K} \equiv \frac{V}{C+V} \]
\[ \implies \frac{1}{Z} \equiv \frac{C}{V}+1 = O+1 \]
Thus our variable $Z$, inverted, while not exactly being directly analogous to $O$, is nearly identical to it, to the point where they might as well be the same. Finally, and most importantly, Marx defined the famous \textbf{rate of exploitation}, otherwise known as the \textbf{rate of surplus value}, to be
\[ E_M = \frac{S}{V} \]
Consider this value in relation to a ratio of the two random variables we just finished defining and discussing in detail: $\frac{R}{Z}$. Note that
\[ \frac{R}{Z} = R\frac{1}{Z} \equiv \frac{S}{C+V}\frac{C+V}{V} = \frac{S}{V} = E_M \]
It appears then that the ratio of the rate of profit to the rate of wage-bill will be crucial to examine in order to examine the relationship between this probabilistic model and Marx's own. We'll call this random variable $X$, and call it \textbf{rate of capital versus labour}:
\[ X = \frac{R}{Z} \]
\\
Next, we turn back to specific price $\Psi$. What we want to look at is large aggregate collections of commodity transactions from market space. Let $\mathbb{B}$ be such an aggregate, chosen uniformly at random. (Note by the notation, we are associating transactions in market space with the commodities being traded, a slight abusse in notation which we will do repeatedly.) Consider the random variable $E(\Psi|\mathbb{B})$, that is, the conditional expectation of $\Psi$ given we are looking at the particular subset of market space which is $\mathbb{B}$. Then
\begin{align}
	E(\Psi|\mathbb{B}) &= \sum_{i: c_i \in \mathbb{B}}\Psi(i)P(i|\mathbb{B}) \\
						&= \sum_{i: c_i \in \mathbb{B}} \Psi(i)\frac{P(i)}{P(\mathbb{B})} \\
						&=\frac{\sum_{i: c_i \in \mathbb{B}} \Psi(i)P(i)}{\sum_{i: c_i \in \mathbb{B}}P(i)} \\
						&= \frac{\sum_{i: c_i \in \mathbb{B}} \frac{\Pi(i)}{\Lambda(i)}\frac{\Lambda(i)}{\sum_{j = 1}^{n_m}\Lambda(i)}}{\sum_{i: c_i \in \mathbb{B}}\frac{\Lambda(i)}{\sum_{j = 1}^{n_m}\Lambda(i)}} \\
						&= \frac{\sum_{i: c_i \in \mathbb{B}} \Pi(i)}{\sum_{i: c_i \in \mathbb{B}} \Lambda(i)}
\end{align}
What this means is that if we were to for whatever reason consider a large random bundle of commodities, then the specific price we would expect to see is the ratio of the total price to the total labour content of that bundle. Note that the definition of the probability measure is crucial to arrive at this. On the one hand, in the algebra here, the details of our probability measure on market space are crucial: the probability of a transaction must be measured by it's weight in labour content for this result to hold. On the other hand, however, the transactions drawn \textit{were completely independent of labour content}, in the sense that the bundle $\mathbb{B}$ is not a drawing of transactions weighted by labour content. It is simply a bundle chosen completely at random. The same result would hold if we were relativizing our model to some other central commodity other than labour power, such as oil. It is a property of the specific price, by it's definition, interacting with the distribution of the central commodity, by it's own definition. \par 
The significance of this result is the following: Suppose we were to draw a random sample of specific prices from market space, $\Psi_1,\Psi_2,...,\Psi_n$, where each $\Psi_i$ is independent and identically distributed. Then by the law of large numbers, the sample mean converges in probability to $E(\Psi) = \mu_{\Psi}$
\[ \bar{\Psi}_n = \frac{1}{n}\sum_{i =1}^n \Psi_i \overset{p}{\to} \mu_{\Psi} \]
That is to say, the probability that the sample mean differs from $\mu_{\Psi}$ by any fixed distance at all goes to $0$ as $n \to \infty$. The more larger our sample is, the closer we can expect the average will be to the actual mean of the distribution. Now, drawing a random sample of specific prices is very similar to drawing a random bundle $\mathbb{B}$, and considering the associated specific prices of commodities in that set. However there are two slight differences. The first, less important difference is the fact that in the latter case, we are implicitly sampling without replacement, whereas in the first case we could conceivably sample the same transaction twice. This difference is minor in the sense that presumably market space is extraordinarily large, and the chance of sampling the same commodity twice is thus quite small. However, it could become more of an issue once considering the second difference, namely the fact that in the random sample case, we are presumably sampling transactions weighted by their labour content. If $50$ percent of all of the total labour content of society goes into a single one of these transactions, then that transaction has a very high chance of being sampled twice, which could produce big differences. \par 
Regardless, if we allow our sample $\mathbb{B}$ to grow larger and larger, it will obviously be the case that $E(\Psi|\mathbb{B})$ converges in probability to $E(\Psi)$. The reason for this is that as $\mathbb{B}$ approaches a larger and larger unbiased portion of the entire space $\mathcal{M}$, $P(i|\mathbb{B})$ must converge closer and closer to simply $P(i)$, so that equation (21) becomes identical to $E(\Psi)$. Thus we have the following crucial result:
\begin{fact}
	For a random and unbiased bundle of transactions $\mathbb{B}$, sufficiently large,
	\[ \frac{\sum_{i: c_i \in \mathbb{B}} \Pi(i)}{\sum_{i: c_i \in \mathbb{B}} \Lambda(i)} \approx E(\Psi) \]
This approximation approaches equality as $\mathbb{B}$ approaches the entire space $\mathcal{M}$ in number of elements. The practicality of this fact, i.e. the ability to use this approximation confidently, given a large and reasonably unbiased bundle of commodities, is contingent on 
\begin{center}
	\textbf{Assumption}: No single transaction holds a particularly large share of the total distribution of labour content.  
\end{center}
\end{fact}
This result gives us a statistical version of Marx's key assumption in Capital Volume 1, recall Marx claimed the existence of a $\psi_0$ such that for any commodity $C$, 
	\[ \frac{\bar{\pi}(C)}{\lambda(C)} = \psi_0 \]
Now, our model has no notion of ideal prices, nor can it predict anything similar to this for a particular commodity $C$. However, consider a large and reasonably unbiased \textit{collection} of commodities $\mathbb{B}$. Assume all commodities in the bundle are sold at some point, and let $\pi(\mathbb{B})$ be the total price of these commodities. Also let $\lambda(\mathbb{B})$ denote the total labour content. Now, our model assumes a fixed period of time $T$, such that $\Pi$ and $\Lambda$ only apply to the commodities bought and sold during this period, long enough to be significant but short enough that labor contents of commodities can be assumed fixed and unchanging. There's no guaruntee that everything in the bundle was produced during this period or bought/sold during this period. However, we can at least reasonably \textbf{assume that the products of the bundle were produced, bought and sold during a period of time \textit{nearby} the "current" period, and assuming nothing massive has changed in the economy in the timespan including these extra periods}, it can then be further assumed that all of the commodities in $\mathbb{B}$, whenever they were bought and sold, have similar price distributions to the current market space, and that consumption patterns more or less repeat themselves. Define the set $\mathbb{B}'$ to be the concrete subset of market space consisting of all commodities in $\mathbb{B}$ sold during the present period, along with, for each commodity in $\mathbb{B}$ not yet in $\mathbb{B}'$, the average of all specific prices of transactions for an identical commodity, if at least one exists (if no similar transaction occurs, simply ignore it). Under our assumptions then  
\[ \pi(\mathbb{B}) \approx \Pi(\mathbb{B}') \] 
and
\[ \lambda(\mathbb{B}) \approx \Lambda(\mathbb{B}') \]
But 
\[ \frac{\Pi(\mathbb{B}'}{\Lambda(\mathbb{B}')} \approx E(\Psi) \]
and so we have our analog to Marx:
\begin{align}
	\frac{\pi(\mathbb{B})}{\lambda(\mathbb{B})} \approx E(\Psi)
\end{align}
Compared to Marx's claim, we have replaced any mention of ideal prices with the actual, observed prices, value with our operationalized notion of labour content, and $\psi_0$ with the new constant $E(\Psi)$. Note the mathematical definiteness of this. We made a few assumptions to get here - what were they?
\begin{itemize}
	\item[(1)] No single transaction during a period $T$ embodies a particularly high percentage of the total labour done during that period.
	\item[(2)] For any unbiased and large collection of commodities $\mathbb{B}$, they were all produced, bought and sold during a period nearby the current period (i.e. no significant portion of commodities in the bundle were produced in, say, the 1980s, if the current year is 2000)
	\item[(3)] The distributions of $\Lambda$ and $\Pi$ change relatively slowly over time. 
\end{itemize}
These are all extremely innocent assumptions to make about our own capitalist economy, and indeed any capitalist economy. Assuming this, the "labour theory of value", in the sense described by equation (26), is not even something that requires empirical evidence to confirm. It is simply a logical tautology, and it would hold if we relativized our model to any other commodity measure satisfying assumptions $(1)$ through $(3)$, whether it be oil, watches, or designer clothing. \par 
Now, let's connect this with the rate of profit. Recall that the general rate of profit for a capitalist society is 
\[ r_g = \frac{\pi(\mathbb{P}) - \pi(\mathbb{I}) - \pi(\mathbb{V})}{\pi(\mathbb{K})} = E(R) \]
Where $\mathbb{P}$, $\mathbb{I}$, $\mathbb{V}$, and $\mathbb{K}$ are the total bundles of produced goods, raw materials, real wages, and capital respectively, of the entire economy. Now, each of these is obviously an \textit{enormous} collection of commodities. That each of them is \textit{unbiased} sufficiently to apply our "labour theory of value" approximation to it is not obvious, and warrants a discussion. F and J make an excellent and very convincing argument in the affirmative (though the argument would not necessarily apply in Marx's own time), which I will paraphrase here later if I have the energy. Assuming this, we can approximate $\pi(\mathbb{P})$ with $E(\Psi)\lambda(\mathbb{P})$, and so forth, and replace each term in $r_g$ with it's labour counterpart. Immediately the $E(\Psi)$'s all cancel out, and we obtain 
\[ E(R) = r_g \approx \frac{\lambda(\mathbb{P}) - \lambda(\mathbb{I}) - \lambda(\mathbb{V})}{\lambda(\mathbb{K})} \] 
Thus, the general rate of profit, which is also the mean rate of profit of society, can be assumed to approximately equal the rate of profit, can be expressed in terms of labour times \textit{interchangeably} with actual money prices.    
However, this entire argument, it must be again and again re-emphasized, would hold for \textit{ANY} commodity satisfying assumptions $(1)$ through $(3)$ above. 

